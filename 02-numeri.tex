\label{numeri}
\chapter{Numeri}

\em{Per tre punti non allineati passa una ed una sola retta, purche' abbastanza spessa (Giulio Cesare Barozzi)}

\epigraph{Per tre punti non allineati passa una ed una sola retta, purche' abbastanza spessa}{Giulio Cesare Barozzi}

%The memoir document class offers "out of the box" commands \chapterprecis, \chapterprecishere, and \chapterprecistoc:
%https://tex.stackexchange.com/questions/53377/inspirational-quote-at-start-of-chapter
%\chapterprecishere{Per tre punti non allineati passa una ed una sola retta, purche' abbastanza spessa\par\raggedleft--- 
%  \textup{Giulio Cesare Barozzi}, Lezioni di Analisi III
%}

In questo capitolo si parler\'a della teoria dei numeri (nella quale son sempre stato poco ferrato): vi dir\'o lo stretto necessario per capire i numeri naturali,
reali, complessi senza la pretesa di dare rigorose specifiche formali degli stessi.

I numeri fondamentali che servono nella matematica sono naturali, interi, razionali, reali e complessi. Cercher\'o di darne una rudimentale definizione,
di fare qualche esempio, e di vederne alcune propriet\'a.

\section{Insiemi notevoli, in breve}

Cominciamo con una carrellata degl'insiemi notevoli (gruppi di numeri con cui ci piace giocare) ed alcune loro proprieta', particolarmente il loro comportamento all'infinito:

* $\mathbb{N}$: Questo e' l'insieme dei \textbf{Naturali}, ovvero {0,1,2,3, .. 42, 100, 1000, 1000000, ... , e tanti altri }. Quelli che in
  italiano chiamiamo numeri interi in matematica si chiamano numeri naturali.

* $\mathbb{Z}$: L'insieme degl'\textbf{Interi} e' simile al precedente ma include anche i numeri negativi. 
 E' grande esattamente il doppio di $\mathbb{N}$, eppure e' grande uguale ad $\mathbb{N}$ \footnote{Eh gia', strana cosa gli infiniti. Si puo' dimostrare che la cardinalita' di N e Z sono uguali, e le sorprese non sono finite.. }.
 I numeri che contiene sono $0,1,-1,42,-42,1000,-1000$, e cosi' via.

* $\mathbb{Q}$: Questo e' l'insieme dei \textbf{Razionali}, ovvero l'insieme dei numeri frazionari espriminbili come $\frac{p}{q}$.
Alcuni esempi sono: $1,2/5,3/4,-42$ . Da notare che i razionali contengono gli interi ma contengono moltissimi piu' numeri.
E' un insieme molto denso nel senso
che tra i suoi 2 valori $1/10$ e $2/10$ ci sono infiniti valori. La cosa forse piu' incredibile e' che (Peano lo dimostro' per primo)
la cardinalita' di $\mathbb{Q}$ e la stessa di $\mathbb{N}$. Lo so, assurdo. Eppur vero. L'infinita' di questi interi e infiniti ad essa
isomorfi e' di difficile comprensione, ve lo concedo.

* $\mathbb{R}$: insieme dei \textbf{reali}. Chiameremo $\mathbb{R}+$ l'insieme dei reali >= 0 e con $\mathbb{R}-$ i reali con x<=0, e con $\mathbb{R}*$ l'insieme dei reali $senza lo \{0\}$.

* $C$: insieme dei \textbf{complessi}. I complessi meritano un capitolo a parte. L'insieme \egrave isomorfo a $R^2$ (ovveri tutti i punti in un piano).

* $\mathbb{R}^3$: insieme di terne di reali. In matematica non esiste ma in fisica si usa un casino per definire la posizione di un punto nello spazio
e ci tenevo a contestualizzarlo qui.

* $\mathbb{Z}_p$: insieme di di interi modulo $p$. Questo insieme si dice anello, e gode di interessantissime proprieta' ma e' difficile parlarne se non decidiamo $p$. 
  Se $p$ e' un numero primo, gode di ulteriori proprieta' ad esempio la possibilita' di inversione per la moltiplicazione. Esempio: $\mathbb{Z}_5 := {1,2,3,4,5}$ o ancora
  meglio $\mathbb{Z}_5 := {0, 1,2,3,4}$. Si', la cosa incredibible di questo insieme e' che 0 e 5 sono la stessa cosa, quindi intercambiabili. Un professore bastardo
  potrebbe addirittura dirvi che $\mathbb{Z}_5 := {420, -4, 427, -42, 420004}$. Se dividete un numero per 5 e guardate al resto, 1 e 6 e 66 sono la stessa cosa, no? 
  Per maggiori informazioni: \href{https://it.wikipedia.org/wiki/Aritmetica_modulare}{Aritmetica modulare su Wikipedia}.

\section{I numeri naturali}

Tutto nasce dai naturali. Schiere di filosofi e matematici hanno provato a darne una definizione. Peano e Russell (se ben ricordo) ne hanno dato una definizione operativa che vi propongo.

- Esiste un numero iniziale che chiameremo $zero$ ($0$).

- Per ogni numero naturale $N$ esiste un successore, che chiameremo $s(N)$ (\'e una funzione, se ci si pensa).\footnote{Quindi $1=s(0),2=s(s(0))$, e cos\'i via. 
No, non vi invito a scrivere 1000 in questa notazione.}.

- Mi pare vi siano altre propriet\'a, ma non mi vengono in mente.

Ora, si pu\'o tranquillamente fare matematica in queso modo, per esempio il famoso $2+2=4$ diventa $s(s(0))+s(s(0))=s(s(s(s(0))))$. Capite pero' che per scrivere $12*12=144$ servirebbero tante righe e per $1'000*1'000=1'000'000$ 
servirebbero intere pagine. Ecco perch\'e la notazione 'posizionale' che tutti conosciamo (10=dieci, 100=cento, 110=centodieci) torna utile per risparmiare inchiostro. Quando capirete che la matematica sfrutta ogni possibile strada
per risparmiare inchiostro, comincerete a pensare come dei veri matematici...

Una interessante propriet\'a dei naturali \'e che non hanno un limite destro, per cui si dice che vanno all'infinito. Vi sono anche studi su $quanti$ siano i naturali.
L'infinit\'a dei naturali viene detta $\aleph_0$, ed \'e la pi\'u piccola infinit\'a conosciuta. Ebbene s\'i, ci sono tanti infiniti, alcuni pi\'u grandi, alcuni pi\'u piccoli.

Denoteremo con $\insieme{N}$ l'insieme dei naturali: $\insieme{N}\cdot\{0,1,2,3,4\ldots\}$


\section{I numeri interi}

Tutti pensano che gl'interi siano i naturali. In matematica, invece, i naturali sono i numeri interi "positivi", men tre si chiamano $interi$ i numeri interi positivi o negativi.
Credo che la definizione sia:

- Esiste lo zero ($0$).

- Per ogni $z$ esiste un successore $s(z)$.

- Per ogni numero $z$ esiste l'opposto $-z$.

L'insieme che viene fuori \'e: $\insieme{Z} \cdot \{0,1,-1,2,-2,3,-3,\ldots\}$. Attenti, potevo anche dire che l'insieme \'e: $\{\ldots,-3,-2,-1,0,1,2,3,\ldots\}$.


\section{I numeri reali}

I numeri reali sono espressi come numeri che possono essere definiti come limite di una successione di razionali. E' davvero buffo pensare che sommando ingredienti tutti razionali dal primo all'ultimo (e potrete immaginare che la somma di due razionali sia sempre razionale!) possa dare come risultato qualcosa che \em{ non} sia razionale.


\section{I numeri complessi}

\b{Introduzione}.
I numeri complessi sono numeri molto interessanti perch\'e ci si \'e arrivati ben dopo agli altri numeri e ci si \'e accorti della loro esistenza a causa di "buchi" nella matematica comune 
che non potevano essere spiegati in altro modo. La genesi di questi numeri, a quanto ne so, nasce dalla necessita' di trovare soluzioni a equazioni polinomiali di grado $N$. ci si \'e infatti
accorti che la soluzione di un'equazione polinomiale di grado 3 (ad esempio) e' spesso data da 3 numeri, ma talvolta da uno solo. E le soluzioni di un'equazione di grado 7 possono essere 7,5,3 oppure 1.
Se ci inventiamo questo mondo "fittizio" dove facciamo finta che la radice quadrata di $-9$ abbia un senso, ecco che quasi per magia si scopre che in questo mondo favoloso le soluzioni di un'equazione
di grado N sono sempre N! Quindi restate con me se la cosa vi incuriosisce, come incuriosisce me.
 

