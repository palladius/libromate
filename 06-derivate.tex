\label{derivate}
\chapter{Derivate}

Questo capitolo tenter\'a di spiegare il concetto di derivata che \'e tanto importante per tutta l'analisi matematica.
Esso esige come prerequisito una buona conoscenza delle funzioni come concetto e delle funzioni notevoli.

\section{Cos' \egrave una derivata}

La derivata \'e una specie di 'lastra' della vostra funzione: \'e un aiuto che vi consente di comprendere meglio la
funzione che state studiando. Buffamente, \'e essa stessa una funzione; mentre la vostra $f(x)$ \'e qualcosa che associa
ad ogni $x$ una $y$ che voi - o il vostro prof - ritenete interessante in quanto tale, la derivata (che chiameremo $f'$)
fornisce invece una ulteriore informazione su $f$ stessa: non tanto il valore di $f$ nel punto (per questa informazione
basta $f$!!!) quanto la sua {\em variazione} l\'a intorno a $x$. Ma facciamo un passo indietro.


%Definizione ANDAZZO
\label{andazzo}
\begin{definizione}[Andazzo] Definiamo \bf{andazzo} di una funzione la pendenza media che ha una curva $f(x)$ sull'intervallo 
      $[a,b]$ (possibilmente continua sull'intervallo stesso); poich\'e l'ho inventato io, lo definir\'a in maniera pignola; esso
      \'e un operatore che accetta in ingresso una funzione $f$ e due numeri $a;b \in \mathcal{D}_f$: $\mathcal{A}nd[f(x);a;b] \doteq \frac{f(b)-f(a)}{b-a}$.
      Per comodit\'a assumeremo $b>a$. Il suo significato pratico \'e di dire quant'\'e
       l'andamento medio di una funzione su un certo intervallo.
\end{definizione}

Vediamo di sfruttare questa definizione per qualche uso pratico. Supponiamo che abbiate investito il vostro danaro (diciamo $10$ \EUR).
A gennaio investite quei $10$ e nei mesi successivi mi trovate i seguenti soldi: $(10,11,10,9,9,10,9,12,13,14,11,11)$. Come potete notare,
a gennaio dell'anno dopo avete guadagnato soltanto $1$ \EUR, mentre qualche mese prima vi eravate illusi di guadagnare ben di pi\'u. Prima di
parlare di derivate, vediamo una cosa: quanto vale $f$? Ci\'o \'e tutt'altro che banale. Poich\'e vi ho dato soltanto 12 punti, la cosa pi\'u
corretta da dire \'e che $f$ ha come dominio $12$ punti ($= \{1,2,3,\cdot,12\}$) e come codominio i valori dati. Un altro approccio \'e di dire
che la funzione \'e definita per tutti i giorni dell'anno, o addirittura per tutti i minuti dell'anno (basta vedere l'andamento di borsa dei
vostri soldi minuto per minuto), ma di tutti questi valori son noti solo i dodici valori mensili. In generale, qualunque sia $f$, pensiamo
che $f$ passi per dodici punti noti: $(1;10),(2;11),(3;10),\cdots,(12,11)$. Ai matematici questa definizione puntiforme di solito piace poco.
Esistono infinite funzioni passanti per quesi dodici punti, ma una delle pi\'u semplici \'e sicuramente (senza usare i $sinc$, per chi li
conosce) il polinomio di grado $12$ che passi per quei punti, che chiameremo $P(x)$.

Ora vediamo di rispondere a qualche domanda: qual \'e stato l'andamento del nostro titolo su base annua? Qual \'e stato l'andamento del
nostro titolo su tutto il primo mese? E su tutto l'ultimo mese? E quale l'andamento alle 14:30 ddell'8 Agosto? Per rispondere alle prime
tre domande \'e sufficiente la definizione di andazzo, per l'ultima \'e opportuna la definizione di derivata; ma dato che gli esempi
pratici sono molto difficili da usare in matematica, torneremo a un esempio assolutamente poco veritiero, innaturale e incomprensibile,
ma la cui derivata sia facile da calcolare :).

Supponiamo che il vostro conto in banca valga $x(t)=2t^2+3t+10$ \footnote{Non spaventatevi: qui la variabile \'e il tempo mentre la
vostra $f$ ora si chiama $x$.
Scusate, ma \'e corretto pure cos\'i e dovrete pure abituarvi no? D'ora in poi $f$ si chiama 'ics' e $x$ si chiama 't'. Lo so, sono cattivo!}.
All'istante zero ($t_0$), in cui depositate i soldi, depositate 10 (per comodit\'a pensiamo che siano migliaia di euro, ma potrebbero essere
anche dollari altairiani). Per non rischiare di dire qualcosa di sensato, diciamo che l'unit\'a di $t$ sia il secondo; tanto per capirci,
dopo 10 secondi che avete depositato i vostri soldi avrete gi\'a maturato da $10$ di partenza $240$ mila euro. Spero ci\'o vi persuada che
siamo in un mondo diverso da quello reale.

Poniamoci un paio di domande, e per ciascuna di esse chiediamoci di che strumenti abbiamo bisogno per rispondere. 

{\em (1) Quanti soldi avremo tra un minuto? (2) Tra quanto tempo avremo 1000 soldi (una specie del nostro 'primo miliardo')? (3) Tra quanto
tempo raddoppieremo i nostri soldi? (4) Quant'\'e l'andamento del mio conto nei primi dieci minuti? (5) Ma \'e conveniente investire in questa banca?
Se s\'i, quando lo \'e e quando no?}

Mettiamoci con carta e penna a risolvere le domande.

(1) Abbiamo $x(t)$, ed \'e pi\'u che sufficiente a rispondere; anzi direi che \'e nata proprio per lo scopo:
"Quanti soldi abbiamo al tempo $t$?". La risposta \'e $x(60)=7200+180+10=7390$. Tantini, direi...

(2) Qua dovremmo usare la funzione inversa: \'e il tempo $t_0$ in cui $x(t_0)=1000$. Calcoliamolo:
$1000=2t_o^2+3t+10 \longrightarrow t_0=\frac{-3 \pm \sqrt{3^2-4 \cdot 2 \cdot (-990)}}{4}$. Delle due soluzioni,
entrambe sensate, prenderei quella positiva dato che per tempi minori di zero abbiamo l'imbarazzo di avere un conto
altissimo in banca senza ancora aver depositato nulla: prodigi della matematica! Queste cose capitano spesso quando
crcate di modellare la realt\'a con dei numeri; i numeri vi aiutano in tanti casi, ma spesso sono 'stupidi' e non
danno per scontato quello che {\em voi} date per scontato. Io vi ho avvisati...

(3) Questa domanda \'e una tipica domanda in cui buttate i dati in ua equazione e sperate che venga fuori qualcosa
di sensato. Ci chiediamo il tempo $t_1$ in cui i nostri $10$ son diventati $20$, ovvero il tempo in cui semplicemente
noi avremo $20$, no? Ecco qua: $20=2t_1^2+3t_1+10$. Anche qui vengono fuori due soluzioni, di cui vi consiglio la positiva.
Non c'interessa calcolarla, solo capire che c'\'e e che significato ha... 

(4) Cos'\'e l'andamento? E' l'aumento (o diminuzione) del nostro capitale in un certo lasso di tempo. Il valore medio
   \'e quindi $\frac{\Delta x}{\Delta t}$. Ma allora la definizione di andazzo (\defpagdef{andazzo}) capita a puntino!
   Proviamo: $\mathcal{A}nd[x(t);0;600]=\frac{x(600)-x(0)}{600-0} = \frac{2\cdot600^2+3\cdot600+10-10}{600}=600\frac{2\cdot600+3}{1} = 600\cdot1203 = 721800$.
   E senza calcolatrice! Qual \'e il significato pratico della risposta? Che mediamente sui primi 10 minuti si pu\'o dire che guadagniamo una media di $721800$ \EUR
   al secondo. E infatti, se moltiplichiamo per $10'$ otteniamo proprio la nostra differenza di capitale tra ora e $10'$ da ora. E' importante notare due cose: primo,
   l'andazzo \'e diverso per intervalli diversi ma di stessa durata (se non ci credete calcolate l'andazzo tra $10'$ e $20'$, scommetto che sar\'a molto pi\'u grande);
   secondo, questo \'e un valore medio che 'smussa' i valori locali in favore di uno studio 'globale'\footnote{Se nel bel mezzo dell'intervallo perdiamo tutto,
   e dopo qualche tempo riguadagnamo tutto l'operatore $\andazzo$ non se ne accorgerebbe: \'e in virt\'u di questa pecca che i matematici hanno inventato la derivata}.

(5) Vediamo di riformulare la domanda: in quali istanti i soldi crescono e in quali i soldi calano? Se potessi calcolare questa cosa, potrei andare in banca appena
  prima di ogni ribasso e riportarglieli quando tornan su!!! Sarebbe una previsione molto interessante, no? Una errore comune \'e confondere la crescita dei soldi
  col segno della funzione. Questo \'e sbagliatissimo. Il segno di $x(t)$ ci dice se siamo in attivo o in passivo. Ma la crescita o la decrescita del denaro ci \'e
  dato dalla derivata, che quindi ci dice quanto ci dobbiamo aspettare che cambi il nostro patrimonio nei brevissimi istanti successivi. Perch\'e brevissimi? Vediamolo
  con la definizione.

\begin{definizione}[Derivata]
Sia data una funzione $f(x)$ continua sull'intervallo $[a;b]$. Sia $x_0$ un punto {\em interno} a tale intervallo ($x_0 \in ]a;b[$). Definiamo la derivata di $f$ nel punto $x_0$ come il valore:
 \begin{equation}
  \lim_{x \rightarrow x_0} \andazzo[f;x_0;x] \doteq f'(x_0)
 \end{equation}
 Poich\'e molti veterani potrebbero essere schifati dal mio operatore fatto in casa (l'Andazzo) vi do ua definizione che non lo usa:
 \begin{equation}
  \lim_{x \rightarrow x_0} \frac{f(x)-f(x_0)}{x-x_0} \doteq f'(x_0)
 \end{equation}
 Notiamo intanto che la derivata non sembra essere una funzione, ma un semplice numero. Tranquilli, non vi ho ingannato.
 In realt\'a, per ogni $x_0$ su cui \'e definita $f$ \'e definita anche $f'$: potremmo definire la {\em funzione} derivata
 come la funzione che segue, punto per punto, i valori dati dalla definizione. Attenzione, per\'o! La derivata non \'e necessariamente
 calcolabile su ogni punto su cui \'e definita $f$: su ogni punto deve esserci abbastanza spazio sia a sinistra che a destra
 affinch\'e il limite possa essere calcolato\footnote{Questo ha a che fare con i cosiddetti {\em punti di accumulazione}, di
 cui non  parler\'o. In generale, \'e sufficiente che esista un insieme $[a,b]$ compatto che racchiuda il punto $x_0$ affinch\'e
 il limite sia calcolabile.}. Un modo spesso pi\'u facile per definire la derivata \'e quella di non dire: "limite per $x$ che
 tende a $x_0$" ma di dire: "limite per $x=x_0+h$ con $h$ che tende a $0$". Se ci pensate non cambia assolutamente nulla, ma i
 calcoli son spesso pi\'u semplici; in tal caso, abbiamo ottenuto non pi\'u $f'(x_0)$ ma gi\'a $f'(x)$:
 \begin{equation}
  \lim_{h \rightarrow 0} \frac{f(x+h)-f(x)}{h} \doteq f'(x)
 \end{equation}
 Il significato fisico pi\'u semplice (e che non dovrete mai dimenticare) della derivata \'e: $f'(x_0)$ \'e sempre il valore
 della {\em pendenza } della retta tangente a $f$ nel punto $x_0$.
\end{definizione}

La derivata \'e, per nostra fortuna, sempre facile da calcolare (ci\'o non vale, ad esempio per gli integrali). Per ogni funzione 'notevole',
esiste una funzione che di essa \'e la derivata. vediamo di ricavarne alcune; delle rimanenti daremo semplicemente la ricetta gi\'a fatta.

\begin{esercizio}
Calcolare la derivata di $x(t)=2t^2+3t+10$, che chiameremo $x'(t)$.
\begin{eqnarray}[rcl]
 x'(t) & = & \lim_{t \longrightarrow h} \frac{f(t+h)-f(t)}{h} =\
       & = & \lim \frac{(2(t+h)^2+3(t+h)+10)-(2t^2+3t+10)}{h} =\
       & = & \lim \frac{2t^2+4ht+2h^2+3t+3h+10-2t^2-3t-10}{h}=\
       & = & \lim \frac{4ht+2h^2+3h}{h}=\
       & = & \lim \frac{(h)(3+4t+2h)}{h}=\
       & = & \lim 3+4t+2h=\
       & = & 3+4t.
\end{eqnarray}
Semplice, no? Notate che le parti che non contenevano $h$ si sono elise a vicenda (riflettete un attimo
sul perch\'e), le parti che la contenevano di grado $1$ (ovvero $(3+4t)h$) sono rimaste fino alla fine,
mentre le parti che la contenevano di grado $2$ sono state elise in quanto infinitesimi di ordine superiore.
\end{esercizio}

Ora possiamo finalmente rispondere alla domanda (5): quanto \'e stiamo guadagnando in un certo momento?
Nel momento in cui versiamo i soldi (tempo $0$), il nostro patrimonio \'e crescente. Esso vale $f'(0)=3$.
Questo vuol dire che la pendenza della retta tangente a $f$ nel punto $0$ \'e $3$, quindi vuol dire che in
un istante {\em piccolissimo} a sinistra e a destra dello zero noi stiamo guadagnando $3$ soldi (nell'esempio,
$3000$ \EUR) al secondo. Attenzione! Non vuol dire come potreste pensare che tra un secondo avrete esattamente
$3$ soldi in pi\'u\footnote{Questo perch\'e in $0$ la funzione $f'$ vale $3$, ma dopo mezzo secondo ha gi\'a
un altro valore, e per sapere quanto avete dovete calcolare l'{\em integrale} di ques'andamento, che banalmente
\'e la $f$ stessa.}, ma qualcosa di comunque simile: vuol dire che tra un secondo avrete {\em circa} $3$ soldi
in pi\'u, tra $\frac{1}{1000}$ di secondo avrete circa $\frac{3}{1000}$ di soldi in pi\'u, tra $\frac{1}{1'000'000}$
di secondo avrete circa $\frac{3}{1'000'000}$ di soldi in pi\'u, e che questi 'circa' si avvicinano a realt\'a
tanto pi\'u quanto pi\'u il tempo diventa piccolo. Vi sembra poco? Se s\'i, vi sbagliate di grosso.

\begin{esercizio}
Calcolare la derivata di $f(x)=x^n$ (con $n \in \insieme{N}^+$).

Per risolvere questa, occorre conoscere i binomi di Newton. Per essi rimando a \refpagref{binominewton}. Vi ricordo che:

\begin{equation}
(a+b)^n=\binom{n}{0}a^n+\binom{n}{1}a^{n-1}b^1+\binom{n}{2}a^{n-2}b^2+\cdots+\binom{n}{n-1}a^1b^{n-1}+\binom{n}{n}b^n,
\end{equation}
Sapendo ci\'o, usiamo la definizione di derivata:
\begin{eqnarray}
D[x^n]  & = & \lim_{h \rightarrow 0}\frac{(x+h)^n-x^n}{h} = \
	& = & \lim \frac{
	(x^n+\binom{n}{1}x^{n-1}h+\binom{n}{2}x^{n-2}h^2+\cdots+h^n) - (x^n)
	}{h} \
	& = & \lim \frac{
	(\binom{n}{1}x^{n-1}h+\binom{n}{2}x^{n-2}h^2+\cdots+h^n)}{h} \
	& = & \binom{n}{1}x^{n-1}\
	& = & nx^{n-1}
\end{eqnarray}
Anche qui, i termini di ordine $h^2$ o superiori sono morti, lasciandoci qualcosa di pulitissimo! La cosa pi\'u bella di questo
risultato \'e che ora sapete derivare ogni singolo pezzo di un polinomio, quindi in pratica sapete derivare un polinomio (non \'e
ovvio, dovete ringraziare il principio di multilinearit\'a della derivata che trovate in seguito).
\end{esercizio}

\begin{esercizio}
Provate a calcolare la derivata di $e^x$, $\log(x)$, $\sin(x)$. Attenti perch\'e il limite \'e fattibile ma richiede una
buona 'cultura' supplementare ;)
\end{esercizio}

\section{Derivate notevoli}

\begin{tabular}{|r|l|}
\hline
Funzione & Derivata\
\hline
\hline
$x^n$ & $nx^{n-1}$ \
$ax^n+bx+c$ & $2ax+b^\dagger$ \
$\frac{1}{x^n}$ & $\frac{-1}{x^{n+1}}^\dagger$ \
$\sqrt{x}$ & $\frac{1}{2\sqrt{x}}^\dagger$ \
$\sqrt[n]{x} (=x^{\frac{1}{n}})$ & $\frac{1}{n\sqrt[n]{x^{n-1}}}^\dagger$ \
\hline
$\log_a(x)$ & $\frac{\log_a(e)}{x}$ \
$\ln(x)$ & $\frac{1}{x} ^\dagger$ \
$\ln|x|$ & $\frac{1}{x} ^\dagger$ \
\hline
$\sin(x)$ & $\cos(x)$ \
$\cos(x)$ & $-\sin(x)$ \
$\tan(x)$ & $\frac{1}{\cos^2(x)} (=1+\tan^2(x))$ \
\hline
$\arcsin(x)$ & $\frac{1}{\sqrt{1-x^2}}$ \
$\arccos(x)$ & $\frac{-1}{\sqrt{1-x^2}}$ \
$\arctan(x)$ & $\frac{1}{x^2+1}$ \
\hline
$a^x$ & $a^x \ln a$ \
$e^x$ & $e^{x \dagger}$\
\hline
$\sinh(x)$ & $\cosh(x)^\dagger$ \
$\cosh(x)$ & $-\sinh(x)^\dagger$ \
$\tanh(x)$ & $\frac{1}{\cosh^2(x)}^\dagger$ \
\hline
$arcsinh(x) (= ln|x + \sqrt{x^2+1}|)$ & $\frac{1}{\sqrt{1-x^2}} ^\dagger$ \
$arccosh(x) (= ln|x \pm \sqrt{x^2-1}|)$ & $\frac{-1}{\sqrt{1-x^2}} ^\dagger$ \
$arctanh(x) (= \frac{1}{2} ln|\frac{1+x}{1-x}|)$ & $\frac{1}{x^2-1} ^\dagger$ \
\hline
\end{tabular}


{\small ($\dagger$): questa derivata pu\'o essere tranquillamente dedotta dalle altre (non \'e 'primitiva'); l'ho messa solo per comodit\'a (leggasi pigrizia) vostra.}

Le derivate pi\'u stupefacenti sono senz'altro quella del logaritmo e quella dell'arcotangente..
ricordatevene quando studierete gl'integrali selle funzioni polinomiali fratte ;)

Ho notato che molta gente ha paura a derivare $\sqrt{x}$ e non dorme al pensiero di derivare $\sqrt[3]{x}$.
Vi ricordo che la radice $n$-ma di qualcosa equivale, in campo reale, a un'esponenziazione: $\sqrt[n]{x} \equiv x^{\frac{1}{n}}$;
ci\'o la riduce a godere delle propriet\'a di derivazione di qualsiasi polinomio!!!


\subsection{Propriet\'a delle derivate}

\subsubsection{Multilinearit\'a della derivata}

$D \alpha f(x)+\beta g(x) = \alpha f'(x) + \beta g'(x)$

% TODO(ricc): hey ricc piu giovane, mi sembra un po forte quest'affermazione forse togli
Tirerei gentilmente un pugno in faccia a chi desse per scontata la relazione seguente, e amabilmente chiederei
a queste persone di dirmi quanto vale la derivata del prodotto di due funzioni... ebbene sappiate che
{\em non vale assolutamente} la seguente: $D f(x)g(x) = f'(x)g'(x)$. Alla faccia delle ovviet\'a.
\subsubsection{Derivata del prodotto}

$D  f(x)g(x) = f'(x)g(x) + f(x) g'(x)$

\subsubsection{Derivata del rapporto}

Questa \'e davvero incasinata:
$D  \frac{f(x)}{g(x)} = \frac{f'g-g'f}{g^2}$

Se mai ve la doveste dimenticare (per esempio \'e facilissimo sbagliare il segno, cosa che succede se al numeratore invertite $f$ con $g$), con un po' di tempo ve la potete ricavare, dopotutto \'e il prodotto di $f$ e $\frac{1}{g}$, e la seconda \'e la funzione composta $\frac{1}{g(x)}=[\frac{1}{x}]\circ[g(x)]$. 

\subsubsection{Derivata di funzione composta}
$D f(g(x)) = f'(g(x))g'(x)$

Attenti, la precedente \'e facilissima da enunciare, ma difficile da capire e applicare. Mi piacerebbe che i pi\'u bravini di voi deducessero dalla precedente equazione la seguente: 

$D \frac{f(x)}{g(x)} = \frac{f'g-g'f}{g^2}$

Oltre questa, potete dedurre anche le seguenti \footnote{Dedurle da voi fa la differenza tra il ricordarle o meno al prossossimo esame:
tutti sappiamo che al prossimo avrete i bigliettini; ma un giorno potrebbe servirvi in un esame in cui non credete che vi possa servire;
\'e {\em l\'i} che potrebbe tornarvi utile. Sono sinceramente convinto che la matematica che non sapete a memoria sia completamente inutile
(ma non mi darete retta poich\'e voi volete solo passare esami, non sapere le cose, e vi capisco perch\'e la pensiamo tutti cos\'i in un
intorno sinistro dell'esame).} (potete dedurle da voi; se avete 5 minuti, investiteli nel dedurre le seguenti):

$D e^{f(x)} = f' \dot e^f$

$D \log^{f(x)} = \frac{f'}{f} $

$D \sin^{f(x)} = f' \dot \cos(f)$

$D \sqrt{f(x)} = \frac{f'}{2\sqrt{f}}$

State molto attenti alla derivata del logaritmo: d'ora in poi quando troverete $\frac{2x}{x^2+42}$ vi
dovr\'a suonare un allarme, solo cos\'i diventerete dei bravi integratori.

\begin{esercizio}
Calcolate le derivate di $\cos(f(x)),\arctan(f(x)),f(x)^x,f(x)^{g(x)}$. Attenti che le ultime due son difficilotte.
\end{esercizio}

\subsubsection{Derivata di funzione inversa}

Anzitutto mettiamo in chiaro una cosa: la funzione inversa {\em non} \'e $\frac{1}{f(x)}$, come molti credono.
E' in realt\'a la funzione che si estrare da $f$ invertendo $x$ e $y$; i grafici delle due funzioni sono speculari rispetto alla retta ($y=x$), se non ci credete guardate il grafico di $e^x$ e della sua inversa $\ln(x)$.

Se io so la derivata di $sin(x)$, non sarebbe bello se potessi usare una scorciatoia per avere la derivata di $\arcsin(x)$? Ebbene, la c
!

$D f^{-1}(x) = \frac{1}{f'(y)} \bigg|_{y=f^{-1}(x)}$


Questo non \'e affatto banale: derivate $f$ e la mettete al denominatore; ora avrete una $g(y)$ che non c'entra pi\'u nulla con $f$. Dovete fare in modo che non dipenda da $x$, $y$ o altro, ma solo da $f$, e ci\'o \'e tutt'altro che semplice. Esempio da manuale:

$D \arcsin(x)= \frac{1}{cos(y)} \bigg|_{y=arcsin(x)}=\frac{1}{\sqrt{1-sin^2(arcsin(x))}}=\frac{1}{\sqrt{1-x^2}}$

Antipatico, vero? Eh gi\'a. Credo meriti un minuto di riflessione: "Quanto fa $\cos(\arcsin(x))$?" Non lo so. Ma so che $\cos = \pm \sqrt{1-sin^2(x)}$. E' un passo avanti, dato che so che $\sin(\arcsin(x))=x$, cos\'i come $pippo(pippo^{-1}(x))=x$ per qualunque funzione $pippo$. C
 un unico appunto da fare: nulla \'e gratis, e abbiamo dovuto pagare un prezzo. Il fatto \'e che abbiamo avuto in risposta $x$ applicando la funzione $seno$ e $arcoseno$, e per far ci\'o ci siam dovuti 'restringere' secondo tutti i restringimenti di dominio apportati dalle funzioni (nel nostro caso, solo la seconda, che restringe $\mathcal{D}=[-1;1]$). Anche il nostro barbatrucco di dire che $\cos(x)=\pm \sqrt{1-sin^2(x)}$ ha avuto un costo; ad esser pignoli dobbiamo dividere il dominio in due parti, a seconda che venga fuori il pi\'u o il meno. Metterei una mano sul fuoco (sperando di non fare scevolate) che alla fine venga ci\'o che ho detto io, ovvero  $\frac{1}{\sqrt{1-x^2}}$. Ma fate i conti per sicurezza.


\section{Massimi e minimi}

Sicuramente, l'aiuto maggiore che possono dare le derivate nello studio di funzione \'e il calcolo di massimi e minimi; poich\'e la derivata \'e la pendenza della retta tangente a $f$ nel punto, gli zeri di $f'$ sono proprio quei punti in cui la funzione smette di crescere o decrescere e si gode un attimo di pace; se passa da positiva a negativa, abbiamo un massimo (cresceva poi decresce), se da negativa diventa positiva abbiamo un minimo; se invece dopo lo zero il segno viene mantenuto \footnote{Vi segnalo l'interessante effetto matrioska: se la derivata intorno a zero passa da positiva a nulla e di nuovo a positiva ha un minimo in zero, e quindi si annulla la derivata seconda!!!} possiamo osservare un flesso, che ai nostri fini \'e solo un falso allarme.

Per maggiori dettagli su derivate, massimi e minimi rimando allo studio di funzione (cap. \refpagref{massimiminimi}).




