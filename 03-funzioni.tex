\label{funzioni}
\chapter{Funzioni}

In questo capitolo si parler\'a di funzioni. Nella prima parte ne verranno dati definizioni ed esempi, 
nella seconda si parler\'a dello 'studio di funzione', e nella terza si affronteranno tanti piccoli temi
collaterali. Ho speso molte parole per spiegarne il concetto poich\'e \'e a mio parere uno dei pi\'u 
affascinanti e complessi della matematica: l'uomo riesce con facilit\'a a ragionare su numeri, ma su qualcosa
che associa infiniti numeri a infiniti numeri tende ad avere sempre delle perplessit\'a. Molto spazio verr\'a
dato alle funzioni notevoli, poich\'e \'e a partire da questi mattoncini che si costruiscono quei maestosi
castelli che i professori mettono nei vostri compiti...

\prerequisito{Dovete guardare prima i grafici.}

\section{Cos'è una funzione}

Cosa è una {\em funzione}? Essa \'e una applicazione che associa tutti 
i valori di un dominio (ovvero un insieme $\mathbf{X}$) a certi valori su un 
codominio $\mathbf{Y}$ (un altro insieme). 
\begin{equation}
f: x \in \mathcal{X} \mapsto \mathcal{Y}
\end{equation}

Detto in parole pi\'u povere, {\em \'e un oggetto 
che mangia numeri e sputa fuori numeri in maniera prevedibile}. Se gli date da 
mangiare una stessa cosa (per esempio $x_0:=46,2$) e lui 'gradisce' questa 
cosa (ovvero $x_0$ è contenuta nel dominio $\mathcal{X}$), la funzione 
sparer\'a sempre fuori uno stesso numero. Esso \'e detto $f(x_0)$. Il concetto di funzione
è molto importante poichè è quanto di pi\'u generale ci sia: una funzione pu\'o 
mangiare qualunque cosa e 'spararla' in quasi qualunque altra cosa. Ci\'o che \'e pi\'u
difficile da capire \'e che il nome della funzione \'e indipendente da quello che le 
si d\'a in 'ingresso'.

Vediamo di inventarci qualche funzione per fissare il concetto.

\begin{equation}
% f(x) \doteq x+3; 
%\end{equation}
%\begin{equation}
 f(0) \doteq 3; \\
 f(1):=4; \\ 
 f(2):=5; \\ 
 f(-5):=-2; \\ 
  \ldots
\end{equation}

La funzione che ho scritto potrebbe essere qualunque cosa; probabilmente la pi\'u
semplice funzione che si comporti come lei \'e: $f_1(x)=x+3$\footnote{Ma attenti,
esistono un bel po' di funzioni che rispettano questi vincoli. E' un po' come cercare
moglie secondi i vincoli: '2 occhi', '2 seni', 'che respiri'. Un altro esempio, solo per persuadere
i pi\'u scettici, \'e $f(x)=4+sen(\frac{\pi}{2}(x-1))$}. Non bisogna sottovalutare 
la \emph{potenza} di questa notazione, per cui prendiamoci un po' di tempo per 
capirla. Quando scrivo $f(x \in \mathbf{R})=x+3$ intendo un oggetto che, dato un numero, 
vi risponde con un altro numero. In ingressso dovete dargli un numero reale e in uscita 
vi restituir\'a un altro numero reale. Per capirci meglio, daremo d'ora in poi a $f$ il 
simpatico nome di '{\bf sommatre}'

Mi piacerebbe spiegarvi le funzioni in un modo alternativo, come me lo spieg\'o il mio prof
di Geometria, Massimo Ferri nel 1995. Pensate di avere un asse orizzontale in cui \'e libero di muoversi un cannone;
esso pu\'o sparare solo in verticale, su o gi\'u, ma sempre in verticale rispetto a dov'\'e. Ora
immaginate che il cannone sia programmato per sparare sempre in un medesimo punto (esempio in $3$)
se si trova in una certa posizione (ad esempio $0$). Se il cannone si trova nel punto $-5$ (che sta
ad ovest, per intenderci), sparer\'a sempre in basso all'altezza di $-2$. Infine, quando si trova
in $-3$, si spara addosso, ma questo non \'e un problema che ci tocchi. Se dopo aver istruito il
cannone su come sparare, lo fate andare all'infinito a destra e a sinistra, potrete osservare la scia
che lascia su un ipotetico foglio di carta; questa \'e il {\em grafico} della funzione. Cos'\'e la 
funzione? L'equazione? Il grafico? Il carroarmato? No! Essa \'e l'insieme di istruzioni che avete dato al carroarmato.
Ci\'o che pi\'u si avvicina alla sua definizione senza perdere infiniti fogli a scrivere $f(0,1,2,3,\ldots)$ \'e
la parola 'sommatre'; ma attenti: se la funzione si complica non sempre avrete la fortuna di poterle dare un nome semplice
quindi dovrete adeguarvi alla notazione usata in quei noiosissimi libri di matematica. Ma prima di arrivare a quella notazione,
cerchiamo di capire il senso dietro ad essa.

E' importante capire che la funzione {\bf sommatre} \'e qualcosa che somma tre al numero 
che le date in pasto. Alcune domande {\em ben poste} potrebbero essere: quanto vale $f(0)$? Per quale
 $x$, $f((x)$ vale 15? Quando \'e massima $f$? Quand'\'e che $f$ incontra gli assi cartesiani? 
 Per quali valori del dominio $\mathcal{X}$  $f$ \'e definita? 
 \footnote{Alcune domande mal poste potrebbero essere: 's\'i ok $f(x)$, ma questa benedetta $x$ quanto cacchio vale?!?'}.
 
Tutte queste sono domande che vi potrebbero essere fatte quando vi si chiede di fare lo \em{studio di funzione} di $f$. 

Mi ritengo soddisfatto se sapete rispondere (per iscritto) a queste domande a trabocchetto: {\bf Che cos'\'e $f(x)$? Che cos'\'e $f(t)$? Che cos'\'e $f(0)$?}  

Una risposta pu\'o essere: le prime due sono \em{funzioni} (per la precisione sono sempre la nostra amica 'sommatre'), la terza \'e un \em{numero} (ed \'e il numero $3$, poich\'e $0+3=3$:
provatelo con le vostre calcolatrici tascabili). Ma allora che cos'\'e $x$? E che differenza c'\'e tra $x$ e $t$? Questa credo sia la domanda pi\'u difficile che ci si possa fare sulle
funzioni. Una buona risposta \'e che la $t$ \'e la $20^{ma}$ lettera dell'alfabeto anglosassone mentre la $x$ \'e la $24^{ma}$.. a parte la battuta il fatto \'e che non c'\'e alcuna differenza
tra $f(x)$ e $f(t)$, se non il fatto che nel primo caso $f$ dipende da $x$ e nel secondo da $t$: la variabile \'e muta di per se stessa,
e ha senso \b{solo} nella definizione stessa della funzione\footnote{La cosa diventerebbe rilevante in qualcosa
come: $f(x)=3tx$, in tal caso $x$ \'e la variabile e $t$ sar\'a probabilmente un parametro.}; ma la funzione sar\'a
sempre lei, nel nostro caso sar\'a sempre 'sommatre', ovvero qualcosa che mangia un numero, gli somma 3 e sputa fuori
il risultato. Un altro modo per capirlo, \'e inventarsi un mondo senza $x$; proviamoci per un attimo. Prendiamo la nota
funzione $f(x)=x^2+2x+1$; prima dell'ottocento, i matematici la leggevano come 'prendi un numero, elevalo al quadrato,
poi aggiungi il doppio del numero e infine aggiungi uno'. Capirete che se un prof vi chiede se questa sia una parabola
o un'iperbole farete abbastanza fatica a rispondergli, no? Allora diremo: $f(numero)=numero^2+2 \cdot numero+1$.
Ancora troppo lungo. Proviamo con la $x$: $f(x)=x^2+2x+1$. Oh! Ora va meglio.

Scusate la lunghezza della digressione, ma dopo anni e anni di matematica, vedo persone ancora perplesse sul vero
significato della fatidica \em{variabile indipendente}. Potete pensare che $x$ sia una variabile \em{muta},
assolutamente inutile, e quindi il vero modo per chiamare la nostra amica 'sommatre' in modo matematico non sia $f(x)$,
bens\'i $f(\dot)$\footnote{credo che l'unico modo di apprezzare una funzione per ci\'o che \'e sia proprio quella di
usare un puntino, del tipo $f(\cdot)=\frac{\cdot+1}{1+\cdot^2}$; ovvio che se usate una $x$ al posto di $\cdot$ si legge
meglio, no?}. Ricordate: se scrivo $f(x)=x+3$ e $g(y)=y+3$, definisco la stessa funzione: $f$ e $g$
sono {\em perfettamente identiche}!!! O se preferite $f(Franca)=Franca+3$ e $g(Iolanda)=Iolanda+3$.
D'ora in poi, per\'o, preferiremo chiamare $x$ ci\'o che la $f()$ mangia e $y$ ci\'o che la $f$ sputa,
e al posto della $x$ useremo $t$ se penseremo a tempi o $\theta$ se pensiamo ad angoli: i matematici ci
tengono molto a dare significati diversi alle diverse lettere, e ne hanno cos\'i tanti che le 26 dell'alfabeto
anglosassone non bastan loro, e devon tirarne fuori anche da quello greco ed ebraico...

Vi faccio un esempio stupido: quando davanti a una birra un amico/a mi dice "Sai Riccardo, mi e' capitato X volte ...." il mio
volto si scurisce perch\'e ho la certezza che la persona davanti a me non abbia studiato o apprezzato la matematica
quanto me. Cos\'i come abbiamo l'archetipo delle Susan e dei Dick in inglese, cos\'i abbiamo l'archetipo delle lettere:

* a,b,c sono parametri/coefficienti
* x,y,z sono variabili reali
* i,j,k,l,m,n sono variabili intere 
* $\alpha , \beta , \gamma$ sono coefficienti pi\'u eleganti di a,b,c. Talvolta usati in congiunzione/parallelo (a vs alpha, ..)
* $\theta, \vartheta, \phi$ sono angoli 

Come avrete capito, l'amico avrebbe dovuto usare N e non X per avere la mia stima :) E s\'i, ovviamente la persona bacata nel cervello sono io, non l'amico. 

\section{Funzioni notevoli}
In questa sezione cercher\'o di analizzare le funzioni pi\'u famose. La maggior parte delle funzioni che vi cappiter\'a di studiare
saranno 'case' costruite con questi mattoncini... Per ciascuna funzione cercher\'o di evidenziare le caratteristiche pi\'u
interessanti di ciascuna, tra cui il grafico (un giorno), derivate, primitive, estremanti, poli, ..

\subsection{Polinomi}

Un polinomio \'e una equazione nella forma:

\begin{equation}
f(x)= \sum_{i=0}^n a_ix^i = a_nx^x+a_{n-1}x^{n-1}+\ldots+a_1x+a_0
\end{equation}

Alcuni esempi di funzione sono: $x^42+37x+41$,$3x^2+6x+3$, $x^2+1$, ma non $\frac{1}{x^2+1}$. Attenzione, anche $3-x$ e $5$
(una delle mie funzioni preferite, l'avrete notato) sono polinomi, e questo la gente tende a dimenticarlo spesso...

Attenzione, come sempre le $x^n$ sono potenze della nostra amica variabile indipendente, mentre le varie $a_0,a_1,\ldots$
sono il DNA del polinomio stesso (non mi credete? Andate a leggervi il capitolo su Taylor, \refpagref{taylor}). Capiamolo con un esempio
stupido: in $x^2+3x+41$ $a_0=41;a_1=3;a_2=1;a_3=0;a_4=0;a_5=0;a_6=0;\ldots$. Se non vi offendete mi fermo qua,
dicendovi che da $a_3$ in poi tutti i coefficienti son nulli. Si dice allora che il polinomio \'e di grado $2$
poich\'e \'e alla posizione $2$\footnote{che \'e al terzo posto, ma ai matematici piace complicarsi la vita,
e spoiler alert! Quasi sempre hanno ragione loro...} che c'\'e l'{\em ultimo} valore non nullo. \footnote{Un modo
molto elegante per definire il polinomio \'e di identificarlo la lista ordinata $(41,3,1)$. Ancora una volta, 
\'e scomparsa la $x$, che guarda caso \'e l'unica cosa non importante per il polinomio!}

\begin{description}
	\item{Dominio} I polinomi non han problemi di dominio: tutto $\insieme{R}$ va bene.
	\item{Codominio} Bazza: se ponete $x=0$ si annulla tutto meno il termine noto, quindi $f(0)=a_0$.
	\item{Zeri} Tanti. In realt\'a ce ne possono essere tanti quanti il grado del polinomio, ma se non avete
	gli 'occhiali' complessi ci possono essere coppie di soluzioni (complesse coniugate) che non potete vedere.
	Esempio: un polinomio di terzo grado ha o $3$ o $1$ soluzioni. Esiste bun trucco per calcolare le soluzioni
	di un polinomio di secondo grado \footnote{Il famoso $x=\frac{-b \pm \sqrt{b^2-4ac}}{2a}$}, un altro per le
	equazioni di terzo e quarto grado, e - se non sbaglio - \'e stato dimostrato che non possono esistere trucchi
	per i polinomi di grado oltre al quarto. {\em Attenti alle molteplicit\'a: $f=x^4+x$ ha 4 zeri: $(0,0,0,-1)$;
	$-1$ \'e uno zero normalissimo, mentre $0$ (che solo per caso si pronuncia allo stesso modo) \'e uno zero di
	molteplicit\'a tripla. Questo \'e molto importante: quando ad esempio una funzione ha uno zero doppio tende
	non solo a passare per il punto ma a esservi tangente (ovvero arrivarci da sotto, toccarlo, e tornar gi\'a
	proprio in quel punto - o viceversa). In generale, non trascurate la molteplicit\'a nei vostri scritti: certe
	persone ci tengono molto ;).}
	\item{Poli} Nessuno.
	\item{Derivata, massimi e minimi} E' molto facile e divertente derivare polinomi. Per la propriet\'a di linearit\'a,
		possiamo derivarla a pezzi, ovvero vale la comodissima propriet\'a che la derivata della somma \'e uguale alla somma
		delle derivate. Se sapete derivare un generico pezzettone $x^n$ (che fa $nx^{n-1}$), sapete derivare tutto. Vien
		fuori: $f'(x)=\sum_{i=1}^n i \cdot a_ix^{i-1} = na_nx^{n-1}+(n-1)a_{n-1}x^{n-2}+\ldots+2a_2x+a_1$. Fate attenzione
		che \'e scomparso il termine noto $a_0$. Il nuovo polinomio ha un grado in meno dell'originale, a meno che non
		avesse grado zero (nel qual caso rimane zero). I massimi/minimi saranno tanti, fino a $n-1$. Calcolate la derivata,
		e di essa calcolate gli zeri. (Io dico sempre: lo studio di funzione di un polinomio \'e per il $60\%$ studio degli
		zeri di $f$, per il $40\%$ studio degli zeri di $f'$: rimane ben poco ancora di interessante).
	\item{Primitiva} Anche qui \'e molto facile, se sapete qual \'e una primitiva dei mattoncini $x^i$ (che \'e $\frac{x^{n+1}}{n+1}$).
		Viene fuori: $F(x)=\sum_{i=0}^n \frac{a_i}{i+1}x^{i+1}+37 = \frac{a_n}{n+1}x^{n+1}+\frac{a_{n-1}}{n}x^{n}+\ldots+\frac{a_1}{2}x^2+a_0x+37$.
		Il $37$ l'ho aggiunto io per ricordarvi che le primitive son tante, e nessuna ha una dignit\'a maggiore delle altre. Se volete usare
		quella col $+0$ anzich\'e quella col $+37$, avete il mio permesso: io farei lo stesso.
\end{description}


\subsection{$\sin(x)$}

La funzione seno (una delle mie preferite) \'e una funzione trigonometrica. Essa \'e periodica, di periodo $2\pi$. Ci\'o
vuol dire che potete disegnarla e studiarla da $0$ a $2\pi$ (ma anche da $7\pi$ a $9\pi$!) perch\'e si ripeter\'a all'infinito con lo stesso
andamento. Ci limiteremo a studiarla su $[0,2\pi]$. Per una definizione pratica di seni e coseni rimando al capitolo su di essi
(\refpagref{trigonometria}). Capirete che una funzione come questa se ha uno zero ne ha infiniti, se ha un polo ne ha infiniti e cos\'i
via... Mi limiter\'o al solo intervallo $[0,2\pi[$. Spesso potrebbe capitarvi di studiiare invece $]-\pi;\pi]$

\begin{schedaf}
	\rigaf{Dominio}{Tutto $\insieme{R}$.}
	\rigaf{Codominio}{E' piccolissimo: $[-1;1]$}
	\rigaf{Zeri}{$0,\pi$}
	\rigaf{Poli}{Nessuno.}
	\rigaf{Derivate}{$D\sin(x)=\cos(x)$. Massimo: $\frac{\pi}{2}$. Minimo: $-\frac{\pi}{2}$}
	\rigaf{Integrali}{$\int\sin(t)dt=k-\cos(x)$. Aree: $A_S[0;\pi]=2$ \footnote{L'area del periodo \'e 4, ma l'integrale \'e nullo: vedi studi di funzione sugli integrali}.}
	\rigaf{Altro}{La funzione e' periodica di periodo $2\pi$.}
	\rigaf{Punti notevoli}{$f(0)=0;f(\pi/2)=1;f(\pi)=0;f(3\pi/2)=-1$.\footnote{Ce n'\'e altri, ma per essi rimando il capitolo ad essi dedicato.}}
\end{schedaf}

\subsection{$\cos(x)$}
Come il seno \'e periodica di periodo $2\pi$: useremo, tanto per allenarci, l'intervallo $]-\pi;\pi]$.
\begin{schedaf}
	\rigaf{Dominio}{Tutto $\insieme{R}$.}
	\rigaf{Codominio}{$[-1;1]$}
	\rigaf{Zeri}{$-\pi/2,\pi/2$}
	\rigaf{Poli}{Nessuno.}
	\rigaf{Derivate}{$D\cos(x)=-\sin(x)$. Massimo: $0$. Minimo: $\pi$.}
	\rigaf{Integrali}{$\int\cos(t)dt=k+\sin(x)$}
	\rigaf{Altro}{La funzione e' periodica di periodo $2\pi$.}
	\rigaf{Punti notevoli}{$f(0)=1;f(\pi/2)=0;f(\pi)=-1;f(-\pi/2)=-0$.}
\end{schedaf}

\subsection{$\tan(x)$}
La tangente altri non \'e che $\frac{\sin(x)}{\cos(x)}$, e ne eredita la periodicit\'a. Anzi, \'e addirittura periodica di periodo $\pi$ (il che non inficia la frase prededente, rifletteteci). La studieremo su $]-\novanta;\novanta]$.
\begin{schedaf}
	\rigaf{Dominio}{$]\-novanta;\novanta[$; \footnote{C'\'e un unico buco in $\novanta$ che significa infiniti buchi a distanza $\pi$!!!}}
	\rigaf{Codominio}{Tutto $\insieme{R}$.}
	\rigaf{Zeri}{$0$}
	\rigaf{Poli}{$\novanta$}
	\rigaf{Derivate}{$D\tan(x)=\frac{1}{\cos^2(x)}=1+\tan^2(x)$ (come preferite). Massimi/minimi: nessuno.}
	\rigaf{Integrali}{$\int\tan(t)dt=k-\log|cos(x)|$}
	\rigaf{Altro}{La funzione e' periodica di periodo $\pi$.}
	\rigaf{Punti notevoli}{$f(0)=0;f(\pi/4)=1;f(\pi/2^-)=+\infty;f(-\pi/2^+)=-\infty$.}
\end{schedaf}


\subsection{Funzioni iperboliche ($\sinh(x)$, $\cosh(x), ..)$}

Le funzioni iperboliche sono fondamentalmente degli esponenziali che puzzano dannatamente di funzioni trigonometriche.
Se li disegnate, o li derivate, sono proprio degli esponenziali. Se li guardate al microscopio (con le serie di Taylor (\refpagref{taylor})),
troverete delle inquietanti somiglianze con i 'cugini' trigonometrici (seno con seno iperboolico, coseno con coseno iperbolico,
e soprattutto tangente e tangente iperbolica). La cosa pi\'u buffa \'e che le funzioni sono completamente diverse (ad esempio,
le iperboliche vanno spesso all'infinito e sono aperiodiche mentre le trigonometriche sono spesso limitate tra $-1$ e $1$ e
sono periodiche). Vediamone la definizione.

\begin{eqnarray}
 \sinh(x) := \frac{e^x-e^{-x}}{2}
 \cosh(x) := \frac{e^x+e^{-x}}{2}
 \tanh(x) := \frac{sinh(x)}{cosh(x)} \bigg( = \frac{e^x-e^{-x}}{e^x+e^{-x}}\bigg)
\end{eqnarray}

Le funzioni inverse di queste tre funzioni iperboliche sono facilmente esprimibili con logaritmi. Riuscite a trovarle?

\begin{esercizio}
 Ricavate la funzione $arcsinh(x)$ e le sue sorelle. {\em Hint: $y=arcsinh(x) \Rightarrow x=\frac{e^y-e^{-y}}{2}, ponete \theta=e^y, \cdots$. Dovrebbe venir fuori qualcosa con logaritmi e radici quadrate mi pare.}
\end{esercizio}

Provate un po' a derivarli: cosa notate di strano? 

\subsection{$e^x$}

Che cos'\'e l'esponenziale ($e^x$)? E' una particolare funzione che viene usata tantissimo in tutti i campi dell'analisi, dell'ingegneria, della fisica; perch\'e?
Credo la sua importanza sia dovuta al fatto che sbuca fuori magicamente dalla Equazioni Differenziali. Prima di andare avanti, vi consiglio di guardare la sezione
sugli esponenti (\sezpagsez{fattoriali}).

Come si calcola un esponenziale? Se l'argomento \'e intero, lo pu\'o fare chiunque sia dotato di addizione e moltiplicazione. Negli altri casi, la faccenda si
complica e richiede anche le radici (quadrate, subiche, ...). Vediamo una defizizione ridondante sufficiente a definire gli esponenziali 'di esponente intero' (poniamo $a \ne 0$):
\begin{eqnarray}
a^0 \doteq 1\\
a^{n+1} \doteq a \cdot a^n\\
a^{-n} \doteq \frac{1}{a^n}
\end{eqnarray}
 
 Da questa definizione, segue che $10^0=1, 10^1=10, 10^2=100, 10^3=1000, \cdots, 10^{-1}=\frac{1}{10},10^{-2}=\frac{1}{100}, \cdots, \ldots$. Come vedete questa funzione cresce in fretta: $f(9)$ vale un miliardo, $f(18)$ vale un miliardo di miliardi e $f(-27)$ vale un miliardesimo di miliardesimo di miliardesimo\footnote{Come si pronuncia $f(-16)$?}. 
 Questa funzione \'e sicuramente la pi\'u veloce di tutte le colleghe (potete immaginare quindi il comportamento della sua funzione inversa, il logaritmo...). Ora vediamo di capire la parte in mezzo
 
Per capire $e^x$ credo dovremo cominciare da qualcosa cui siete pi\'u familiari, tipo la funzione (molto simile) $10^x$ (che si legge $10 alla x$). 

\subsection{$\log(x)$}

TBDs 

Notiamo anzitutto che la derivata di $\ln(x)$ coincide con quella di $\ln|x|$. Se guardate i grafici,
vi accorgete che a destra vale $ln(x)$ e a sx dell'asse y $-\ln(x)$. Qui il valore assoluto pu\'o essere visto come un 'completamento naturale di dominio'.

%\subsection{Funzioni iperboliche?!?}
% skippo va
%tbds

\section{Studi di funzione}

Cosa vuol dire \em{studiare} una funzione? Vuol dire semplicemente dire tante cose interessanti di una funzione.
Supponiamo che il vostro prof di matematica vi porti in centro citt\'a il sabato pomeriggio e vi proponga qualcosa
di simile: lo {\em studio di persona}. Vi addita un passante e vi dice: cosa puoi dirmi di lui? Voi direte:
\'e alto $1,70$ circa, pesa sui $70 kg$, ha una corporatura media, i capelli mori ricci e porta gli occhiali;
ha un neo vicino alle labbra e ha dei seni particolarmente grossi. C'\'e qualcosa di non banale in ci\'o che
avete detto: l'altezza, il peso, il colore dei capelli, degli occhi eccetera sono caratteristiche di \em{qualsiasi}
persona. Gli occhiali, i nei, lentiggini, eccetera sono invece caratteristiche di \em{alcuni} individui, eccezioni
della cui presenza ci si accorge ma la cui assenza passa inosservata. Difficilmente direte: "senza occhiali",
"senza barba", "senza n\'ei visibili", o sbaglio? (Ovvio che alla polizia direte anche queste cose se si tratta di un ricercato!)

Lo studio di funzione \'e qualcosa di molto simile: si tratta di delineare una funzione (come la nostra amica 'sommatre')
secondo alcune caratteristiche comuni a {\em tutte} le funzioni, e di trovare quei tratti della funzione che sono interessanti.

\subsection{Dominio e codominio}

La prima cosa da studiare in una funzione \'e sicuramente il dominio di definizione della funzione stessa, ovvero l'insieme
di valori (le $x$, per intenderci) su cui la funzione \'e definita. Alcuni esempi:

\begin{equazione}
 f_1(x)=3x^4-2x^3+87x-57\pi \\
 f_2(x)=\sin(x) \\
 f_3(x)=\log(x) \\
 f_4(x)=\sqrt(x+1) \\
 f_5(x)=\frac{(x-29)(x+3)}{(x-29)(x-12)(x-1976)} 
\end{equazione}

I casi pi\'u complicati sono certamente i primi due: teniamoli dunque per ultimi. Se conoscete la funzione logaritmo, sapete che essa \'e
definita solo per $x>0$, quindi il $\mathcal{D}_{f_3}=\{\mathbf{R} \senza \mathbf{R}^-\}$. Attenti alla pignoleria: $\mathbf{R}^+$ comprende
lo zero, quindi un modo carino \'e prendere tutto $\mathbf{R}$ e togliergli $\mathbf{R}^-$, liberandomi in un sol colpo dei numeri negativi e dello zero.
Il caso $f_4$ \'e molto simile: la radice quadrata esige un argomento non negativo (ovvero i numeri minori di zero le sono indigesti). Attenti per\'o
all'argomento: non c'\'e $x$ questa volta, ma una funzione {\em molto} pi\'u complicata: $x+1$. Abituatevi a ci\'o: per calcolare il dominio della
funzione dovete imporre che l'argomento sia non negativo, il che produce la seguente equazione: $x+1 \ge 0$. Dunque $\mathcal{D}_{f_4}=\{x|x \ge -1\}$.

Il caso $f_5$ vi capiter\'a spesso. Come sapete dalle elementari, la funzione {\em divisione} non accetta un divisore (che
\'e la parte che sta dopo il diviso) uguale a zero, quindi se avete che la vostra funzione \'e uguale a
$f(x)=\frac{qualcosa}{Pippo \cdot Bluto}$ dovrete imporre che {\em sia} Pippo {\em sia} Bluto siano diversi da zero.
Ci\'o significa, nel nostro caso particolare, imporre $x$ diverso da quei tre numerini presi {\em assolutamente} a caso:
$\mathcal{D}_{f_5} = \mathcal{R} \senza \{12;29;1976\}$. Un appunto: quei tre numeri hanno assolutamente la stessa dignit\'a,
e non ne viene uno prima dell'altro (lo dico perch\'e molta gente pensa il contrario).

Veniamo ora ai primi due: vedete qualche pezzettone di funzione che vi faccia scattare un qualche allarme? Vedete poli, radici,
logaritmi, o cose del genere? No, dunque il dominio delle prime funzioni \'e proprio $\insieme{R}$.

In generale, studiare il dominio di una funzione vuol dire partire da $\insieme{R}$, e poi cominciare a stringerlo per ogni
funzione strana che c'\'e l\'a dentro, introducendo le limitazioni della funzione (logaritmi, equazioni fratte, radici, etc).

Parliamo ora di {\bf codominio}. Il codominio \'e l'insieme di valori che la funzione produce.

\subsection{Incontro con gli assi}

Questa caratteristica \'e una delle pi\'u generiche eppur interessanti: si tratta semplicemente di vedere in quali punti
la nostra $f$ incontra l'asse delle $x$ (detti 'zeri') e l'asse delle $y$ (che non hanno nome che io sappia: li chiameremo
{\em antizeri}). La seconda \'e facilissima, mentre la prima \'e pi\'u complicata. Vediamo, ricordandoci che l'asse $x$
ha come equazione $y=0$ (disegnare per credere!) e viceversa l'asse $y$ ha $x=0$:

\begin{equation}
zeri: \bigg\{
	\begin{array}{l}
	  y=f(x); \\
 	  y=0; 
	\end{array}
\hspace{10pt} anti-zeri: \bigg\{
	\begin{array}{l}
	  y=f(x); \\
 	  x=0; 
	\end{array}
\end{equation}

Nel primo caso otteniamo l'equazione $f(x)=0$, che pone la domanda '{\em Per quali $x$ otteniamo un valore nullo?}':
non necessariamente la soluzione c'\'e, n\'e se c'\'e \'e unica. Esempio 1: $f=x^2+42$ non si annulla mai
(se $x \in \mathbf{R}$), quindi non ha zeri; esempio 2: $f=x^2-81$ ha due zeri, poich\'e si annulla sia per $x=9$
(lo zero pi\'u famoso) che per $x=-9$ (lo zero che amo chiamare '{\em forse non tutti sanno che}', anche qui provare
per credere). Purtroppo non sempre \'e facile ottenere gli zeri di una funzione; un modo per risolvere il problema
\'e quello di invertire la funzione (vedi \ref{funzioneinversa}), ma non aspettatevi di trovare in genere TUTTE le soluzioni. :(

Tutto ci\'o che si pu\'o dire \'e che si ottiene una relazione nella forma $h(x)=0$, dove non sempre \'e facile
estrapolare le $x$ per cui $h$ si azzera. Non saprei darvi dei trucchi, poich\'e la soluzione va vista caso per caso;
se la funzione \'e facile, la soluzione \'e facile; se \'e media, la soluzione di solito \'e abbastanza facile (per
esempio: $f=\frac{sin(x)}{x}$); se \'e molto difficile, la soluzione \'e ancora pi\'u vantaggiosa: nemmeno il vostro
prof la saprebbe risolvere e quindi di certo non ve lo chiede. Come vedete, in ogni caso ce l'abbiamo fatta.

\esempio{La funzione 'sommatre' incontra l'asse delle $x$ nel punto $-3$.}

Il secondo caso \'e talmente facile (non sto scherzando) che la maggior parte degli studenti ci si perde (secondo
il motto 'pi\'u la soluzione \'e vicina pi\'u \'e difficile da vedere'). Esso pone la domanda: '{\em Per quali
$y$ la $x$ vale zero?}' Ma attenzione: noi abbiamo una funzione 'esplicitata' per dirci a quale $x$ corrisponde
quale $y$. E' in generale difficile dire quali $x$ producono $0$ (ad esempio), ma \'e facilissimo dire quale
$y$ \'e prodotta da $0$!!! Quindi, \'e sufficiente mettere $0$ nella scatola magica e vedere cosa viene fuori;
qui abbiamo la certezza che il risultato sar\'a esistente (purch\'e la $x$ data appartenga al dominio) e unico.

\em{Nota}. Molto spesso alle superiori o in esami universitari \'e richiesto il famigerato 'Studio di funzione'. Chiedere a uno studente di prepararsi
a casa per poter sostenere a un esame uno studio di funzione \'e un po' come chiedere a un cuoco di prepararsi a casa a fare piatti per poi
alla prova finale saper realizzare un piatto su richiesta: ci si pu\'o preparare su tante cose diverse ma alla fine ogni funzione ha caratteristiche
rilevanti che la rendono unica. Per alcune, ad esempio, \'e importantissimo calcolare l'asintoto, eppure la maggior parte delle funzioni non ha asintoti! 
Esse possono avere poli, zeri, assi o poli di simmetria, e via dicendo.

\label{massimiminimi}
\subsection{Massimi e minimi}

Un'altra cosa importantissima per qualunque funzione sono i punti in cui questa smette di crescere cominciando a scendere,
e viceversa. Purtroppo per fare questo tipo di studio occorre conoscere derivate (\refpagref{derivate}) e limiti (\refpagref{limiti}):
rimando fortemente a questi capitoli prima di procedere con la lettura. 

Dando ora per scontato che sappiate derivare meglio che allacciarvi le scarpe, dir\'o qualche cosa sulla connessione
tra derivate e massimi (o minimi, che \'e la stessa cosa\footnote{Ai matematici piace parlare cos\'i: se Dio vi d\'a
in mano il modo di tirar fuori un massimo da ogni funzione, non avete bisogno d'altro: per il minimo basta studiare la
funzione '$-f(x)$': meditare, gente.}: quando in una funzione si azzera la derivata \'e {\em molto} probabile che sia un
punto di massimo ($49.5\%$) o minimo ($49.5\%$), ma pu\'o essere anche nessuno dei due ($1\%$). Per saperlo con esattezza,
occorre ispezionare meglio la funzione con le derivate successive; il modo esatto se ben ricordo \'e (e lo percorreremo
con un esempio):

\begin{enumerate}
	\item Prendere la funzione. {\em Ad esempio $f(x)=x^4+2x^3+4$ }
	\item Derivarla e calcolare gli zeri (chiameremo questi valori {\em estremanti}); {\em $f'(x)=4x^3+6x^2$ nell'esempio vien fuori $x_1=0;x_1=0;x_2=-frac{3}{2}$}
		\footnote{Scusate la pignoleria: abbiam detto che la molteplicit\'a degli zeri \'e importante, quindi ve lo rimarco; ma il punto \'e sempre uno e uno solo,
		quindi diamogli un nome solo, no?}
		.
	\item Calcolare la derivata seconda e vedere quanto vale $f''$ per ciascuno degli estremanti: $h_i=f''(x_i)$. {\em $f''(x)=12(x^2+x)$; $f''(x_1)=0$; $f''(x_2)=9$.}
 	\item Per tutti gli estremanti con derivata seconda non nulla, siamo a posto: $f''(x_i)>0$ implica che l'estremante $x_i$ \'e effettivamente un {\em minimo}, mentre
	 	se la derivata \'e minore di zero lo lascio alla vostra immaginazione. Se la derivata \'e nulla \'e un grosso guaio (quell'$1\%$ di cui vi parlavo).
		In tal caso occorre procedere con le derivate successive.\footnote{Trucchetto che uso io per ricordarmi a memoria che $f''>0$ \'e minimo e $f''<0$ \'e massimo:
		prendo la funzione $f(x)=x^2$. E' una parabola che conosco a memoria: ha il culetto nell'oriegine e si protende verso l'alto quindi in $(0;0)$ ha un minimo.
		La derivata prima \'e $2x$ (che ci d\'a ovviamente $x=0$) e derivata seconda $f''=2$ che \'e sempre positiva. Dunque positivo $\Longleftrightarrow$ minimo.
		Spero vi aiuti, ma ora che la scrivo non so quanto sia mnemonica...}
	\item (Caso sfortunato) Per ogni estremante che ha derivata seconda nulla, esso pu\'o essere: flesso ($98\%$), massimo ($1\%$) o minimo ($1\%$). Cosa vuol dire flesso?
		Vuol dire che se tagliate la funzione nelle parti sinistra e destra e date i grafici delle due met\'a a due passanti a caso (possibilmente laureati in matematica),
		uno dei due vi dir\'a che l'avete tagliato intorno a un minimo, l'altro vi dir\'a che l'avete tagliato intorno a un massimo. Per vedere cosa sia, dovete fare la derivata
		terza (tranquilli, se siete sfortunati abbastanza potreste dover andare avanti fino alla derivata centesima). Se \'e non nulla, \'e un flesso e siamo a posto. Se \'e nulla,
		ricominciamo daccapo con la derivata quarta; se non \'e nulla abbiamo massimo o minimo (esattamente come con la derivata seconda); se \'e nulla avanti con la quinta:
		si ripete il caso sfortunato in cui per\'o dovete sommare due agli ordini delle derivate. {\em $f'''(x_1)=24x_1=0$: non \'e un flesso, purtroppo, quindi si va avanti
		con $f''''(x)=24$. Eccoci alla fine del calvario: il punto $(0;4)$ \'e un minimo. Nella mia vita mi \'e capitato una volta un caso come questo, e u altro centinaio
		di volte di potermi fermare alla derivata seconda. Ma dovevo prepararvi al meglio no? Spero di non aver generato confusione.}
\end{enumerate}

\subsubsection{Limiti notevoli}
Se dovete stuudiare una funzione che non \'e definita in un punto (come $\frac{1}{x+1}$) o non \'e definita {\em a partire} da un certo punto (ad esempio $log(x)$) \'e opportuno
che vi calcoliate il limite della funzione nel punto d'interesse; nel primo caso, \'e il punto 'fuori dominio' (nell'esempio $-1$): , nel secondo \'e il primo punto (da sinistra o destra) fuori dominio (nell'esempio $0$).
% \subsubsection{curvatura}
% In rarissimi casi, potrebbe servirvi il raggio di curvatura

\subsection{Aree e integrali}
A volte vi si potrebbe chiedere di calcolare una l'area di una porzione di piano che abbia vagamente a che fare col vostro grafico. In tal caso, l'integrale della funzione
(se non lo sapete calcolare, andatevelo a studiare a \refpagref{integrali}) pu\'o tornarvi molto utile. L'unica cosa da sapere \'e che, se $f(x)$ \'e la vostra funzione
e $F(x)$ \'e una sua primitiva, vale la relazione:
 \begin{equation}
  A_R=F(b)-F(a) \bigg( = \int_a^b f(\xi)d\xi \bigg),
 \end{equation}
dove $A_R$ \'e l'area del rettangoloide delimitato dai seguenti 4 punti: $A \equiv (a;0)$, $B \equiv (b;0)$, $C \equiv (b;f(b))$, $D \equiv (a;f(a))$. Si chiama rettangoloide
poich\'e ha 3 lati belli dritti pi\'u uno completamente curvo; esso \'e l'area del luogo dei punti che stanno {\em in verticale} tra $0$ e $f(x)$ nel percorso orizzontale che
va da $a$ a $b$. 
\footnote{In realt\'a questo non \'e corretto: l'integrale \'e talmente bravo che conta le aree come negative se $f$ va sotto l'asse delle $x$, e positive altrimenti. 
Con ci\'o potreste anche ritrovarvi (e succede spesso, fidatevi) con l'area di due triangoloni - uno sotto l'altro sopra l'asse $x$ - che ammonta a zero! Per evitare questi
errori, dovete spezzare l'integrale in pi\'u pezzi, anzich\'e su $[a,b]$, su $[a,x_1],[x_1,x2],...,[x_n,b]$, dove i vari $x_i$ sono i punti in cui la funzione si annulla, e prendere di ogni area il valore assoluto.}



\section{Altro sulle funzioni}

In questa sezione verranno descritte caratteristiche interessanti sulle funzioni che non potevano essere messe nei capitoli precedenti.

\subsection{Relazioni}

E' brutto definire una funzione senza aver definito una relazione, poich\'e quest'ultima nasce concettualmente prima. Vediamolo con un esempio:
\begin{equation}
x^2+y^3-2xy=0
\end{equation}
Questa equazione definisce un insieme di punti del piano (coppie $(x;y)$). In particolare un punto $P(x_0;y_0)$ appartiene a questa curva se e solo se, mettendo i numerini $x_0$ e $y_0$ nella parte sinistra dell'equazione viene fuori $0$! Esempio: $(0;0)$ fa parte della equazione, mentre $(7,6)$ non appartiene (provare per credere: $49+216-84$ non fa $0$, purtroppo)

\subsection{Funzioni inverse} \label{funzioniinverse}

Data una qualunque funzione $f$, la funzione inversa $g$ di una funzione $f$ \'e una funzione ad essa 'complementare' che ha il seguente scopo: se $f$ mangia $x$ e sputa $y$, $g$ dev'essere in grado di mangiare quella $y$ e sputare di nuovo $x$. Vi assicuro che questo non \'e un compito facile, e infatti invertire una funzione non sempre \'e possibile. Vediamo alcune funzioni:

\begin{equation}
f_1: y=x+3; \hspace{5pt} 
f_2: y=41; \hspace{5pt} 
f_3: y=x^2+1; \hspace{5pt} 
f_4: y=2x;  \hspace{5pt} 
f_5: y=\frac{9}{5}x+32; 
\end{equation}

Diamo intanto un nome alle quattro funzioni per poterne parlare meglio: nell'ordine le chiameremo 'sommatr\'e', 'quarantuno', 'Pippo', 'doppiodi'
e 'c2f'\footnote{La quinta \'e un tipico caso di funzione che ha un'utilit\'a nella vita vera, ma voglio lasciare un'aura di mistero su di essa
e sul perch\'e del nome}. Vediamo: 'doppiodi(8)=16', hmmm... s\'i, direi che torna. A dir il vero avrei potuto chiamare 'Pippo' col nome 'unopiuquadratodi':
ho scelto Pippo cos\'i quando vi troverete a un esame a studiare $f=\frac{e^{x+3}+1}{x^3+\log(x)}$ non vi sentirete costretti a terminare l'inchiostro con
'segnodifrazioneconalnumeratoreesponenzialediicspi\'utreiltuttopi\'uunoealdenomminatorexallatrepi\'ulog\'ics' e potrete usare un simpatico 'Luisa'.

Il modo pi\'u facile per trovare la funzione inversa \'e non ragionarci (se no saremmo uomini, e non macchine), fare un'operazione che in matematica non
ha pari quanto a inutilit\'a ma che aiuta l'uomo per i secoli di preconcetti sui simboli: invertire il simbolo $x$ col simbolo $y$. \footnote{E se non
credete a quanto vi ho detto ditemi che figura geometrica corrisponde all'equazione $y=ax^2+bx+c$ e quale a $b=xa^2+ya+z$... eheheeh.}. 

\begin{equation}
g_1: x=y+3; \hspace{10pt} 
g_2: x=41; \hspace{10pt} 
g_3: x=y^2+1; \hspace{10pt} 
g_4: x=2y; \hspace{10pt} 
g_5: x=\frac{9}{5}y+32; 
\end{equation}


Chiamiamo pseudoinversa la schifezza che viene fuori (dico schifezza poich\'e non necessariamente viene fuori una funzione) ricavando la nuova $y$
(che era la vecchia $x$) in funzione di $x$ (la vecchia $y$). Portiamo come si suol dire 'a sinistra' la $y$:

\begin{equation}
g_1: y=x-3; \hspace{10pt} 
g_2: y=?!?; \hspace{10pt} 
g_3: y=\pm \sqrt{x-1}; \\
g_4: y=\frac{x}{2}; 
g_5: y=\frac{5}{9}(x-32); 
\end{equation}

Ehi, ma che succede? La funzione $g_2$ non ha una $y$: come la porto a sinistra?!?!? E $g_3$? Non mi piace mica tanto: prima per ogni $x$ veniva
fuori qualcosa, ora invece per alcune $x$ ha un valore doppio, per altre non ha proprio valore! Che casino!

Ebbene, questo \'e ci\'o che vien fuori a usare la matematica col 'pilota automatico'. Vediamo di ragionare su ogni funzione. Vediamo intanto
le funzioni 'ben educate': $g_1$ \'e diventata la funzione 'sottraitre', $g_4$ \'e 'met\'adi' e $g_5$ la chiamo $f2c$ (\em{soddiabbolico}, lo so).

Perch\'e $g_2$ e $g_3$ non si comportano bene? La definizione di funzione era semplice: per ogni cosa che mangi devi sparare fuori {\em uno ed
un solo valore} e sempre lui. Se $f_2$ sputa fuori $41$ qualunque cosa le si dia in bocca, allora $g_2$ dovr\'a per forza fare una dieta a base di
$41$, per sua stessa definizione: queso vuol dire che $g_2$ avr\'a come dominio il solo numero $41$ (in matematichese si scrive $\mathcal{D}=\{41\}$)
e come codominio tutto $\mathbf{R}$. Ma questo \'e impossibile, e lo si vede dal fatto che se il vostro cuginetto vi chiede quanto vale $g_2(41)$,
voi non potrete rispondergli se non in maniera zen: essa vale tutto e niente. Tra gl'infiniti punti che $g_2$ tocca, se vogliamo renderla una {\em funzione},
dobbiamo sceglierne uno solo. Propongo $5040$, dato che ha un valore affettivo per me. Ecco dunque che $g_2$, definita come $g_2(x)=5040,x \in \{41\}$, \'e
ora una funzione. Se vi divertite a re-invertirla non otterrete $f_2$, come potreste aspettarvi, ma otterrete qualcosa di vagamente simile: ovvero una
restrizione di $f_2$ ad un dominio che le consenta di essere invertibile (ovvero la sola $x=5040$).

Veniamo ora a $g_3$ che, essendo pi\'u difficile da calcolare, \'e paradossalmente molto pi\'u facile da capire \footnote{Se non ci credete,
chiedete a uno studente universitario di disegnarvi la funzione $sin(x)$ e la funzione $5$: di solito avr\'a meno problemi a disegnare la prima!
Oppure vi chieder\'a se non vi siete sbagliati a scrivere la seconda!}. Siamo d'accordo che non ha valori per $x<1$. Dunque il dominio di $g_3$ \'e
$\mathcal{D_3}=\{x | x \ge 1\}$. Altro problema \'e quello del '$\pm$': \'e necessario infatti battezzare uno dei due, e per farvi capire che
nessuna scelta ha una dignit\'a rispetto all'altra lancer\'a un dado: 1,2,3: pi\'u, 4,5,6: meno. E' uscito 6, quindi scelgo meno. Ecco dunque che
$g_3 \doteq -\sqrt{x-1}$ (con $x \in \mathcal{D_3}$) \'e una funzione inversa di $f_3$.

Cosa abbiamo imparato? Che invertire una funzione $f$ \'e fattibile solo se prima si restringe il dominio di $f$ (a propria scelta) in modo che
essa sia iniettiva (ad ogni elemento e' associato al piu' un elemento del codominio) e suriettiva (ogni elemento del codominio e' coperto).

\subsection{Funzioni di pi\'u variabili}

Le funzioni di cui abbiam parlato finora sono le funzioni di singola variabile. Esse sono le pi\'u interessanti poich\'e si parte da esse per
studiare le altre. Sappiate per\'o che esistono anche funzioni di pi\'u variabili, come ad esempio:

\begin{equation}
 f(x,y) \doteq x^2+3y^2-8
\end{equation}

Esse sono, in generale, funzioni che hanno bisogno di mangiare pi\'u variabili per sputarne fuori {\em una sola}. La pi\'u famosa credo sia
$f(x,y) \doteq x+y$ che in Italia e in altre nazioni viene chiamata {\em somma}, ma essa ha la stessa dignit\'a di $x-y$ (differenza), $x/y$
(rapporto), $x^2+y^2-1$ (circonferenza goniometrica) e $e^{x^2+y^2}$ (collinetta di Gauss). Sinceramente, l'ultima \'e quella che preferisco.
Da notare che in generale $f(x,y) \neq f(x,y)$, e lo potete notare dalla funzione differenza. Poco importa se in quasi tutti i miei esempi ci\'o
vale: chiamatela coincidenza; in realt\'a l'uomo fa fatica a sparare cose {\em davvero} a caso, pure quando ci si impegna. Davvero. 

\subsection{Funzioni composte}

Le funzioni composte sono funzioni di funzioni. Se $f$ e $g$ sono due funzioni e c'\'e compatibilit\'a tra il codominio di $g$ e il dominio di $f$,
allora definiamo $h \doteq f \circ g$ come quella funzione che associa a ogni $x$ il valore $f(g(x))$. Perch\'e c'\'e un problema di dominio?
E' un semplicissimo problema di dieta: se $g$ \'e onnivoro e sputa fuori di tutto, mentre $f$ \'e vegetariana esister\'a probabilmente una $x$
che $g$ trasforma in carne, e $f$ non pu\'o mangiarla. L'unico modo \'e restringere il dominio di $g$ a quelle sole $x$ che - trasformate in
$y$ da $g$ - non possano essere indigeste per $f$. Il vincolo, in matematichese \'e che $\mathcal{Y}_g \in \mathcal{X}_f$. I domini non devono
coincidere: non c'\'e alcun problema, infatti, se $f$ \'e onnivoro ma $g$ gli passa soltanto verdure.

Molti miei conoscenti fanno confusione con le funzioni composte. Vediamo di fare qualche esempio, definendo alcune funzioni e chiamandole per
nome (rispettivamente 'sommauno','cinque','Pippo','raddoppia'):

\begin{equation}
 \begin{array}{lr}
f_1: y=x+1; \hspace{5pt} &	% sommauno
f_2: y=5; \hspace{5pt} \\		% cinque
f_3: y=x^2+1; \hspace{5pt} &	% pippo
f_4: y=2x;  \hspace{5pt} 	% raddoppia
 \end{array}
\end{equation}

Proviamo a giocherellare con le 16 combinazioni possibili, vi va?
\begin{equation}
 \begin{array}{rcll}
f_1(f_1) & = & sommauno(sommauno(x))=x+2; & (sommadue) \\
f_1(f_2) & = & sommauno(cinque(x))=6; & (sei) \\
f_1(f_3) & = & sommauno(Pippo(x))=x^2+2; & (Pluto) \\
f_1(f_4) & = & sommauno(raddoppia(x))=2x+1; & (unopiudoppiodi)\\
f_2(f_1) & = & cinque(sommauno(x))=5; & (cinque)\\
f_2(f_2) & = & cinque(cinque(x))=5; & (cinque)\\
f_2(f_3) & = & cinque(Pippo(x))=5; & (cinque)\\
f_2(f_4) & = & cinque(raddoppia(x))=5; & (cinque)\\
f_3(f_1) & = & Pippo(sommauno(x))=x^2+2x+2; & (Paperino)\\
f_3(f_2) & = & Pippo(cinque(x))=26; &	(ventisei)\\
f_3(f_3) & = & Pippo(Pippo(x))=x^4+2x^2+2; &(Paperone)\\
f_3(f_4) & = & Pippo(raddoppia(x))=4x^2+1; &(Minny)\\
f_4(f_1) & = & raddoppia(sommauno(x))=2x+2; &(duepiudoppiodi)\\
f_4(f_2) & = & raddoppia(cinque(x))=10;& (dieci)\\
f_4(f_3) & = & raddoppia(Pippo(x))=2x^2+2;& (Qui)\\
f_4(f_4) & = & raddoppia(raddoppia(x))=4x;& (quadruplica)
 \end{array}
\end{equation}

Secondo il solito famoso 'principio inverso della semplicit\'a', la funzione 'cinque' \'e quella pi\'u dispotica e pi\'u difficile da usare. Notate che la
funzione meno intuitiva, Pippo, \'e quella responsabile della nascita di un'intera famiglia di funzioni non intuitive (che ho chiamato in modo simile a lui).
Molte persone confondono la 'composizione' di funzioni con il prodotto; non poche volte ho visto porre $f_1(f_1(x))=(x+1)^2$. Se date un nome alle funzioni
come faccio io, questi errori  dovrebbero capitarvi di rado. {\em Si noti che in generale $f \circ g \neq g \circ f$}.

\begin{esercizio}
Cosa risulta dalla composizione delle funzioni 'sommauno' ($x+1$) e 'togliuno' ($x-1$)? E dalla composizione di 'raddoppia' ($2x$) e 'dimezza' ($\frac{1}{2}$)?
Qual \'e l'elemento neutro della composizione?
\end{esercizio}
