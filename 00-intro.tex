\label{introduzione}
\setcounter{chapter}{-1}
% cosi sono a capitolo 0
\chapter{Introduzione}

%\citazioneinizioparagrafo{I'm writing a book. I've got the page numbers done.}{Steven Wright}

\citazioneinizioparagrafo{
    Mathematics, rightly viewed, possesses not only truth, but supreme beauty—a beauty cold and austere, 
    like that of sculpture, without appeal to any part of our weaker nature, without the gorgeous trappings 
    of painting or music, yet sublimely pure, and capable of a stern perfection such as only the greatest art 
    can show.
    }{Bertrand Russell}

\includegraphics[height = 300pt, width = 300pt]{images/RiccardoMatterhorn.jpeg}

Se vi trovaste su un'isola deserta per qualche anno, una volta messi a posto i bisogni primari, e foste alla ricerca di un hobby, non potreste dedicarvi a ricercare la storia
perche' vi mancherebbero i libri e i dati per studiarla, e correlare eventi; non potreste dedicarvi alla geografia. Potreste probabilmente dedicarvi alla Fisica, ma vi fermereste
abbastanza presto a riscoprire Newton e presumo non potreste andare oltre un certo punto per mancanza di un laboratorio ben equipaggiato; potreste inventare una pila, una doccia,
e sicuramente vi sarebbe utile. Se vi passasse per la mente di studiare la matematica, tutto cio' che vi serve e' un bastoncino e tanta sabbia. L'indomani probabilmente dovreste
riscrivere alcuni passi spazzati via dal vento, ma le conquiste importanti (ad esempio, supponiamo che ieri abbiate riscoperto Pitagora) vi rimarrebbero impresse in mente.
A me personalmente piace giocare con la matematica inventando un problema e cercando di risolverlo con carta e penna, senza altri aiuti - come in un'isola deserta. Ho sviluppato 
per questo motivo un approccio inquisitivo (ma perche' facciamo cosi'? E' davvero l'unico modo di fare cosi' o i Marziani potrebbero farlo diversamente? E se avessi mo 6 dita
invece di 5, quanto varrebbe Pi greco? E il teorema di pitagora varrebbe anche su Marte o su una sfera?).

In questo libro cerco di non dirvi pedissequamente cos'e' un limite o una funzione, ma cerco di spiegare con esempi che - si spera - possano avvicinare persone che fino a ieri 
hanno ODIATO la matematica, magari dissacrando quegli altarini di "aulicita'" che magari ve l'hanno resa antipatica sulla crosta. La matematica per me e' come una noce di cocco: ci 
vuole un cacciavite per aprirla, ed e' difficile la prima volta, ma l'interno e' succoso e pieno di sorprese. Il mio scopo e' fornirvi il cacciavite e mostrarvi quanto succoso sia il contenuto.

\section{Per chi e' questo libro}

Questo libro e' inteso per due categorie di persone: 

1. Persone con la terza media (o piu') che abbiano completato la parte piu' elementare della matematica (somme, sottrazioni, percentuali, qualche funzione a sistema) e siano
curiose di conoscere concetti un po' piu' complessi.

2. Universitari che sono costretti ad affrontare questi concetti complessi.

Cerchero' di insegnare la matematica con toni giocosi, talvolta volgari (e slang bolognese), con l'arrogante ipotesi che questo linguaggio possa essere piu' vicino ai giovani,
e possa insegnare la matematica divertendo. Passero' particolare tempo a spiegare non tanto COSA sia un certo concetto\footnote{Persone piu' brave di me avranno fatto un ottimo
lavoro e saranno disponibili online, se volete definizioni esatte e teoremi ben enunciati.} ma perche' lo studiamo, perche' e' importante, e tutte quelle piccole cose che i
vostri professori hanno probabilmente dato per scontato.

In altre parole cerchero' di fare per la matematica quello che un bravo professore di storia cerca di fare per la Storia: fornirvi il contesto.

E visti gli ultimi tempi con il Covid-19, mi sono accorto di quanto ci sia bisogno di matematica (e statistica) nella vita di tutti per evitare di dire le baggianate che sto ascoltando nei 
social Network. :)

\b{Nota}. Questo "libro" e' stato scritto a internet spenta (probabilmente alla prossima passata mi accertero' di non aver scritto baggianate).
