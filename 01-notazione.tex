\label{notazione}
\chapter{Notazione}

In questo capitoletto tratteremo il difficile problema di capirci: non solo quando si parla italiano (il che e' lasciato
alla mia capacita' di scriverlo e a quella delle vostre maestre delle elementarii), ma anche quando si parla matematichese
(con quegli strani simboli come $\forall$, $\nu$, $\doteq$, $|$, \ldots). Questo capitolo vuole in parte spiegare come i
matematici parlano matematichese (80\% circa) e in parte come {\em io} parlo matematichese (il rimanente 20\%).
Tenetelo in considerazione almeno ogni tanto; ora come ora, vi do il permesso di saltarlo, poiche' e' certamente il piu' noioso.

\section{Insiemi famosi}

Insiemi notevoli:

* $\mathbb{N}$: Questo e' l'insieme dei \textbf{Naturali}, ovvero {0,1,2,3, .. 42, 100, 1000, 1000000, ... , e tanti altri }. Quelli che in
  italiano chiamiamo numeri interi in matematica si chiamano numeri naturali.

* $\mathbb{Z}$: L'insieme degl'\textbf{Interi} e' simile al precedente ma include anche i numeri negativi. 
 E' grande esattamente il doppio di $\mathbb{N}$, eppure e' grande uguale ad $\mathbb{N}$ \footnote{Eh gia', strana cosa gli infiniti. Si puo' dimostrare che la cardinalita' di N e Z sono uguali, e le sorprese non sono finite.. }.
 I numeri che contiene sono $0,1,-1,42,-42,1000,-1000$, e cosi' via.

* $\mathbb{Q}$: Questo e' l'insieme dei \textbf{Razionali}, ovvero l'insieme dei numeri frazionari espriminbili come $\frac{p}{q}$.
Alcuni esempi sono: $1,2/5,3/4,-42$ . Da notare che i razionali contengono gli interi ma contengono moltissimi piu' numeri.
E' un insieme molto denso nel senso
che tra i suoi 2 valori $1/10$ e $2/10$ ci sono infiniti valori. La cosa forse piu' incredibile e' che (Peano lo dimostro' per primo)
la cardinalita' di $\mathbb{Q}$ e la stessa di $\mathbb{N}$. Lo so, assurdo. Eppur vero. L'infinita' di questi interi e infiniti ad essa
isomorfi e' di difficile comprensione, ve lo concedo.

* $\mathbb{R}$: insieme dei \textbf{reali}. Chiameremo $\mathbb{R}+$ l'insieme dei reali >= 0 e con $\mathbb{R}-$ i reali con x<=0, e con $\mathbb{R}*$ l'insieme dei reali $senza lo \{0\}$.

* $C$: insieme dei \textbf{complessi}. I complessi meritano un capitolo a parte. L'insieme \egrave isomorfo a $R^2$ (ovveri tutti i punti in un piano).

* $\mathbb{R}^3$: insieme di terne di reali. In matematica non esiste ma in fisica si usa un casino per definire la posizione di un punto nello spazio
e ci tenevo a contestualizzarlo qui.

\subsection{simboli}

\begin{enumerate}
  \item{$\doteq$} Simbolo di definizione; il $90\%$ delle volte si intende: definisco ci\'o che sta a {\em sinistra} con ci\'o che sta a {\em destra}. Ad esempio: $ x \item{$\doteq$} 2$ vuol dire: definiamo X come il numero 2 (facile),
  mentre con $ f(x) \item{$\doteq$} x$ intendiamo: definiamo come $f$ la funzione che e' uguale a $x$ per ogni $x$, ovvero la funzione identita' (molto piu' coomplesso). Nel primo caso associamo a un acino d'uva il valore 2, nel secondo
  associamo a un grappolo d'uva tanti valori diversi che dipendono da che acino sia. 
  \item{$\equiv$} Simbolo di equivalenza; $a \equiv b$ vuole dire che $a$ \'e equivalente a $b$ (el $90\%$ delle volte si intende: definisco ci\'o che sta a {\em sinistra}
  con ci\'o che sta a {\em destra}). E' qualcosa di pi\'u forte dell'uguaglianza (so che fa strano, ma pensate a dire $f(x)=g(x)$ e a dire $f(x) \equiv g(x)$: nel primo caso
  cerchiamo di vedere per quali $x$ $f$ e $g$ si ugagliano, nel secondo stiamo dicendo che le due funzioni sono perfettamente identiche).
\end{enumerate}

\section{Come leggere una funzione}

\subsection{Una funzione facile: una retta}

Cominciamo con un esempio facile. Proviamo adesso a leggere insieme una formula "complessa" e cercare di sfatare i miti.

\begin{equation}
  y = a x + b 
\end{equation}

Oppure (piu esplicita): 

\begin{equation}
  f(x) = a x + b 
\end{equation}

Come leggerla: la funzione $f$ o $y$ e' funzione della variabile X. Quanto vale? Beh, cambiando a casaccio X e assegnadogli valori
come 0, 1, 42, 1000, .. la funzione vi dice che il suo valore sara', rispettivamente: $b$, $a+b$, $42a+b$ e $1000a+b$.

La parte piu' complessa di questa operazione e' capire il ruolo di tutte quelle lettere, per cui faro' uno sforzo a dirlo in modo semplice e chiaro:

* X e' la variabile, e in quanto tale puo' far cio' che vuole. La funzione "itera" 
  (o meglio, si muove) al variare di X e quindi X e' libera di pascolare dove e come meglio crede.
* $a,b$ sono invece dei coefficienti noti. Sono un po' ipocriti, lo ammetto, perche' il matematico sa che i valori sono fissi
  ma invece di dire che valgono, ad esempio, 2 e -3, diciamo semplicamente $a,b$. C'e' un motivo aldila' del sadismo puro:
  in genere se facciamo cosi' e' perche' l'andamento concettuale della funzione non cambia molto al cambiare di a,b. Lo capiremo
  meglio durante lo studio di funzione.

\subsection{Una funzione piu' complessa: un generico polinomio}

Questa e' abbastanza criptica:

\begin{equation}
  f(x)= \sum_{i=0}^n a_ix^i 
\end{equation}

% = a_nx^x+a_{n-1}x^{n-1}+\ldots+a_1x+a_0

Essa va letta come: somma per i (parametro libero di spaziare su N valori) che va da $0$ a $n$ di $a_i$ che moltiplica $x$ alla $i$.

La parte a mio avviso piu' difficile qui e' capire cosa e' un numero e cosa e' una variabile.

Come sempre, siamo in una funzione di $x$ quindi capiamo dal lato a sinistra dell'uguale che la variabile e' la $x$.
E il resto? Cosa sono quei pedici? 
Ebbene, ricordate il caso facile $ a x + b $? Bene, ora iniziamo una discussione virtuale col vostro professore:

- Bene, ora estendila ad una generica equazione di grado due
- Voi: facile, basta aggiungere un X quadro, con un coefficiente a raglio davanti:

\begin{equation}
  f(x)= a x + b + c x^2
\end{equation}

Si pero' e' bruttina (ha i coefficienti a raglio, a,b,c moltiplicano rispettivamente frado 1,0,2, non va bene),
permutiamo i coefficienti e rendiamola piu' leggibile:

\begin{equation}
  f(x)= a x^2 + b x + c
\end{equation}

Molto meglio! Ovvio che in questo caso i coefficienti a,b della retta sono ora b,c, ma poco male.

- Bene, bravo, ora generalizzala ad un'equazione di grado tre:

- Nulla di piu' facile prof, ormai ho capito:

\begin{equation}
  f(x)= a x^3 + b x^2 + cx + d
\end{equation}

- Bene, bravissimo! Ora generalizzala ad un'equazione di grado N:

- Uhm, N? che significa N? Dieci? Di piu'? Beh cosi' a occhio se N fosse anche 100 verrebbe fuori qualcosa tipo:

\begin{equation}
  f(x)= a x^100 + b x^99 + ... + v x + z
\end{equation}

Ora ho un problem amolto piccolo, le lettere sono solo 21 (26 se scomodiamo le anglosassoni), ma io ho 101 coefficienti qui,
come faccio? Beh i matematici hanno risolto questo problema con un trucco: invece di scomodare una lettera diversa $a,b,c$ 
usiamo una sola lettera $a$ e la iteriamo da 1 a 100: $ a_1, a_2, a_3, ... a_100 $. Facile no? Possiamo addirittura anche
assegnare $ a_0, a_{-1} $ alla bisogna, purche' siano iterabili (quindi della cardinalita' di $\mathbb{N}$ , non di $\mathbb{R}$). 

Bene, allora un modo piu' elegante di esprimere questa notazione e': 

\begin{equation}
  f(x)= a_100 x^100 + a_99 x^99 + ... + a_1 x + a_0
\end{equation}

Ora, se voleessimo fare i pignoli e togliere quei puntini di sospensione che fanno molto "il mio completo elegante e'
in lavatrice quindi sono venuto a fare l'esame in tuta", potremmo scomodare una sommatoria e fare un'ottima impressione: 

\begin{equation}
  f(x)= \sum_{i=0,100}^100 a_i x^i 
\end{equation}

Da notare che qui abbiamo due "variabili": la X che e' la  nostra cara variabile di funzione e la $i$. Si noti che la $i$
serve solo per "sgirellare" la sommatoria, vuol dire: hey tu, si', proprio tu: prendi un generico $i$ che va da 0 a 100 e 
srotolalo al mio segnale. 

* Quando $i$ vale 0, sostituiscila e scrivi quel che vien fuori: $a_0 x^0$ (convenzionalmente $1$), quindi  $a_0$.
  e quindi $a_0$. Wow, questo era strano - vediamo se von gli altri e' piu' facile: 
* Quando $i$ vale 1, $a_1 x^1$ ovvero $a_1 x$. Ok, facile.
* Quando $i$ vale 2, $a_2 x^2$ ovvero $a_2 x^2$. Ok, facile.
* Potrei andare avanti cosi' da 3 a 99 ma non lo faro'.
* Quando $i$ vale 100, $a_100 x^100$ ovvero $a_100 x^100$. Ok, facile. 
* Ora capisco. Mettendo tutti i pezzi insieme vengono: 

\begin{equation}
  f(x)= a_0 + a_1 x + a_2 x^2 + ... + a_100 x^100
\end{equation}

il che e' la versione in tuta del piu' elegante: 

\begin{equation}
  f(x)= \sum_{i=0,100}^100 a_i x^i 
\end{equation}

Ed eccoci da capo.

Come vedete, quando i matematici scrivono queste equazioni criptiche, non lo fanno per sfoggiare (ok, forse un po'..),
lo fanno soprattutto per rendere un concetto complesso con un piccolo numero di caratteri, che definisce \em{precisamente}
quel che vogliono definire. Questa notazione e' fondamentale per esprimere concetti complessi. Voi potreste pensare: ma si'
dai Riccardo, siamo onesti, potevi usare i 3 puntini invece della sommatoria, volevi solo fare lo sborone. In verita' la
risposta corretta e': si', in questo caso particolare i 3 puntini erano sufficienti a far capire questa sommatoria, ma come
avrei potuto spiegare qualcosa di complesso come una sommatoria tripla? Con 3 serie di 3 puntini? E chi itera su chi? Con 
piu' somme, ci saremmo tutti confusi e la versione con $\varSigma$ sarebbe l'unica con spiegazione univoca.
