\label{taylor}
\chapter{Serie di Taylor}

Questo capitolo tratta l'interessantissimo - a mio vedere - mondo delle serie di Taylor. A cosa servono?
Boh, direi che la serie di Taylor \'e un modo 'pixelloso' di vedere una cosa 'curva', o un modo semplice
di vedere una cosa complessa. Purtroppo, per capirne i concetti dovrete avere buone basi di serie/successioni,
derivate e funzioni. Vi anticipo subito che Taylor \'e il mio argomento preferito, ed \'e per farlo capire
al mondo che ho deciso di scrivere 'sto libro. Se avrete capito questo capitolo e saprete calcolare Taylor di $sin(x)$
alla fine di questa lettura, il mio lavoro si potr\'a dire concluso e sar\'o una persona molto felice (per favore
fatemelo sapere!)

\begin{prerequisito}
1. Quanto vale la derivata $n$-ma di $\sin(x)$? 
2. Quanto vale la serie $\sum_{n=0}^\infty \frac{1}{n!}$? 
3. Quanto vale {\em in $0$} la derivata $k$-ma di $e^x$? E di $\sin(x)$? E della nostra amica $ax^2+bx+c$?
\end{prerequisito}

Se non sapete rispondere a questi prerequisiti, vi consiglio di rileggervi i capitoli Limiti e Derivate, e ripassare di qui.
%refpagref{limiti})  (TODO).

\section{Lo sviluppo di Taylor}

Taylor e' una specie di Raggi X di una funzione, o meglio ancora una TAC: mentre con una fotocopiatrice potete fare solo una copia della superficie di un oggetto (di solito un foglio A4),
mentre nella  \href{https://it.wikipedia.org/wiki/Tomografia_computerizzata}{TAC} potete fare tante (N) fotocopie,
a diverse profondita' e quindi fare una scansione 3D di un oggetto (il vostro cervello, reni, milza, ..), data dalla giustapposizione
concettuale di tutti questi strati. Taylor fa la stessa identica cosa con una dimensione in meno (quindi e' una TAC semplificata). 

Cominciamo con una definizione che probabilmente pochi capiranno, poi faremo un esempio facile, taylorizzando una funzione che non ha bisogno, e infine una funzione un po' piu' difficile e dovreste poi vedere la luce. 

\begin{definizione}[Taylor] Definiamo operatore Taylor di ordine $n$ data una funzione $f(x)$ su un punto $x_0$ (detto anche "punto di microscopio")
    $T_n[f(x), x_0] := \sigma_0^n a_k \frac{(x-x_0)^k}{k!}$, dove 
    $a_k := f^{(k)}(x_0)$ (per costruzione)
\end{definizione}

In altre parole, il polinomio di Taylor di grado $N$ (se non capite, pensate a 10) e' definito come quel polinomio di grado N (10) che passa
per dove passa la funzione nel punto $x_0$ (punto dove mettete il vostro ipotetico microscopio), e ha anche la derivata prima uguale alla funzione 
nel punto dove avete messo il microscopio, e cosi' via con la derivata seconda, fino alla n-ma. 
Ora per costruzione questo polinomio "fantoccio" si coimporta esattamente come la funzione in $x_o$: ha lo stesso valore, la stessa pendenza, varianza, curtosi, asimmetria (skewness) e cosi' via.
Stiamo costruendo un sosia che nel punto di osservazione e' il piu' simile possibile alla funzione, ma allontanandoci dal punto ci sara' probabilmente una divergenza (a meno che le due funzioni si eguaglino). 
Ora se fate abbastanza derivate (1000? Un milione?), il polinomio artefatto e pixelloso che avete costruito aderira' sempre di piu' alla funzione anche allontanandoci da $x_0$.

D'ora in poi considereremo $x_0 = 0$ senza perdere in generalita' (spostiamo l'asse delle ordinate sul punto dove volete mettere il microscopio). A questo punto Taylor si semplifica notevolmente in:



\section{Esempi}

Esempio 1 idiota: polinomio.

1. Prendiamo la funzione: $f(x) = x^2 + 5$. Questo e' un polinomio di secondo grado e sappiamo che abbiamo solo 2 derivate al pi\'u non nulle.
L'ho scelta perch\'e cos\'i non facciamo notte, e poi passiamo ad una pi\'u bella e interessante.






2. Prendiamo ora $f(x) = cos(x)$.




3. Caso generico.