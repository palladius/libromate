\label{taylor}
\chapter{Serie di Taylor}

Questo capitolo tratta l'interessantissimo - a mio vedere - mondo delle serie di Taylor. A cosa servono?
Boh, direi che la serie di Taylor \'e un modo 'pixelloso' di vedere una cosa 'curva', o un modo semplice
di vedere una cosa complessa. Purtroppo, per capirne i concetti dovrete avere buone basi di serie/successioni,
derivate e funzioni. Vi anticipo subito che Taylor \'e il mio argomento preferito, ed \'e per farlo capire
al mondo che ho deciso di scrivere sto libro. Se avrete capito questo capitolo e saprete calcolare Taylor di $sin(x)$
alla fine di questa lettura, il mio lavoro si potr\'a dire concluso e sar\'o una persona molto felice (per favor
fatemelo sapere!)

\begin{prerequisito}
Quanto vale la derivata $n$-ma di $\sin(x)$? Quanto vale la serie $\sum_{n=0}^\infty \frac{1}{n!}$? Quanto vale {\em in $0$} la derivata $k$-ma di $e^x$? E di $\sin(x)$? E della nostra amica $ax^2+bx+c$?
\end{prerequisito}


\section{Lo sviluppo di Taylor}

Taylor e' una specie di Raggi X di una funzione, o meglio ancora una TAC: mentre con una fotocopiatrice potete fare solo una copia della superficie di un oggetto (di solito un foglio A4),
mentre nella  \href{https://it.wikipedia.org/wiki/Tomografia_computerizzata}{TAC} potete fare tante (N) fotocopie,
a diverse profondita' e quindi fare una scansione 3D di un oggetto (il vostro cervello, reni, milza, ..), data dalla giustapposizione
concettuale di tutti questi strati. Taylor fa la stessa identica cosa con una dimensione in meno (quindi e' una TAC semplificata). 

Cominciamo da un esempio idiota, taylorizzando una funzione che non ha bisogno, ma questo pu\'o servire. 

% esempio 1 idiota: polinomio.

1. Prendiamo la funzione: $f(x) = x^2 + 5$. Questo e' un polinomio di secondo grado e sappiamo che abbiamo solo 2 derivate al pi\'u non nulle.
L'ho scelta perch\'e cos\'i non facciamo notte, e poi passiamo ad una pi\'u bella e interessante.






2. Prendiamo ora $f(x) = cos(x)$.




3. Caso generico.