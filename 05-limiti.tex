\label{limiti}
\chapter{Limiti}

Questo paragrafo spiega il difficile (almeno a mio parere) e anti-intuitivo concetto di {\em limite}. Esso è propedeutico al
concetto di derivata ed integrale, per esempio, e consiglio di familiarizzare coi limiti prima di procedere con argomenti che li usano.

\section{Limiti: un'introduzione}

I limiti sono a mio parere uno dei più potenti e interessanti costrutti umani per l'analisi matematica; 
con ciò voglio dire che grazie ad essi si va molto lontano, ma sono anche convinto che se degli extraterrestri 
inventassero un loro sistema di matematica non potrebbero vivere senza interi, reali, derivate e funzioni questi potrebbero 
però benissimo andare avanti senza la nostra definizione di limite. E' una mia impressione: trattatela come tale.

Vediamo di iniziare con un esempio. Prendiamo la funzione:

\begin{equation}
 f(x)=\frac{x^2-4}{x-2} \hspace{20pt} \bigg( = \frac{(x+2)(x-2)}{x-2} \bigg)
\end{equation}

Attenti! Non barate! So che vi verrà la tentazione di dire che $f(x)=x+2$, ma a essere esatti la cosa non vale. Se facciamo
un piccolo studio di funzione, ci persuadiamo subito che la funzione {\em non} è definita per $x=2$, mentre in ogni altro punto
essa vale $f(x)=x+2$ (il che dimostra che non avete ancora completamente dato di matto). Questo capitolo serve a creare una base
teorica per poter dire la seguente, rilassante frase: "La funzione $f$ non è definita in $2$, ma se la differenza tra $x$ e $2$ è
piccola, $f$ si avvicina molto a $4$". Sembra banale ma, come diceva il mio professore di Analisi III, "Sono 50 anni che faccio
matematica e nessuno mi ha mai spiegato cosa sia un numero piccolo". Anzitutto provare per credere (calcolatrice alla mano, se vi serve):

\begin{equation}
\begin{array}{rcl}
 f(1.9) &=&3.9 \\
 f(1.99) &=& 3.99 \\
 f(1.9999) &=& 3.9999 \\
 f(2) & = & ?!? \\
 f(2.0001) &=& 4.0001 \\
 f(2.01) &=& 4.01 
\end{array}
\end{equation}

La tentazione che mi viene è di mettere un cerotto a $f$ e creare una funzione simile (chiamiamola $\stackrel{g}{f}$ o $f_{rep}$)
che viene 'riparata' nel punto di discontinuità (nel nostro caso la $f_{rep}$ varrebbe proprio $x+2$, esattamente come vi sareste
aspettati). Questo procedimento si applica per analisi di Fourier, dove di fatto la serie di Fourier di una funzione e' di fatto
la funzione "cerottata", ma questo va oltre lo scopo di questo capitolo.

\section{Limiti: definizione}

Diremo che:

\begin{equation}
\lim_{x \to x_0} f(x) = y_{0}
\end{equation}
{\em (che si legge 'limite per $x$ che tende a $x_0$ di $f(x)$ è $y_0$')} se vale la seguente condizione:
\begin{equation}\label{definizionelimite}
\forall \epsilon >0 \exists \nu_\epsilon : \{ |x-x_0|<\nu_\epsilon \Longrightarrow  |f(x)-y_0|<\epsilon   \}
\end{equation}

Lo so, lo so, suona assolutamente incomprensibile; vediamo di capirlo; anzitutto chiamiamo $P_{incriminato} \doteq (x_0;y_0)$.
Detto ciò, potete asserire che il limite di $f$ in $x_0$ è $y_0$ con assoluta certezza solo se, comunque un antipatico (che
chiameremo Tommaso poiché non si fida di noi) fissi una $\epsilon$ 'piccola a piacere', voi siete in grado di trovare una
$\nu$ in funzione di $\epsilon$ tale che se la funzione dista in orizzantale da $P$ meno di quanto avete detto voi, essa dista
pure in verticale meno di quel che vi ha detto Tommaso. In altre parole, $\epsilon$ è il $\Delta Y$ massimo che Tommaso accetterà,
e $\nu_\epsilon$ è il $\Delta X$ massimo, dettato da voi, intorno al quale la funzione si comporterà così bene da non uscire dai
limiti (che brutta parola!) imposti da Tommaso.

Notiamo anzitutto una cosa: la definizione di limite non aiuta a {\em scoprire} quanto un limite valga, bensì a {\em verificare}
se esso tenda effettivamente a un certo valore. Ovvero: se un uccellino vi dice che $\lim_{x \to 5} f(x)=7$, potete verificare
se l'uccellino abbia ragione o torto; ma questa definizione non vi aiuta assolutamente a capire quanto sia $\lim_{x \to 5} f(x)$:
se $7$,$8$,$9$, o $-\pi$.

Vediamo di usare questa definizione per verificare la funzione definita all'inizio del capitolo (\refpagref{equazionefratta}).
Anzitutto concorederete con me che calcolare il limite in un punto diverso da $2$ sia possibile ma alquanto inutile: non serve
la potenza dei limiti dato che la funzione fuori da $2$ è definita in modo semplice.

\begin{equation}
 \lim_{x \to 2} \frac{x^2-4}{x-2}=4
\end{equation}
Vediamo se è vero applicando la definizione:
\begin{equazione}
\forall \epsilon>0 \exists \nu_\epsilon : \{ |x-2|<\nu_\epsilon \Longrightarrow  |\frac{x^2-4}{x-2}-4|<\epsilon   \}\\
|\frac{x^2-4-4(x-2)}{x-2}|<\epsilon \longrightarrow |\frac{x^2-4x+4}{x-2}|<\epsilon\\
|\frac{x^2-4x+4}{x-2}|<\epsilon \longrightarrow |\frac{(x-2)^2}{x-2}|<\epsilon\\
|x-2|<\epsilon
\end{equazione}

Ora che abbiamo semplificato, è il turno di Tommaso a fissare un $\epsilon$ (errore massimo voluto): diciamo che lui dica $0.1$
(un decimo). Non è gentile con noi, poteva dire $10$, ma dobbiamo abituarci a tali bassezze se vogliamo andare in giro orgogliosi
a dire che il limite vale $4$! Ora noi dobbiamo trovare un $\nu_\epsilon$ che faccia valere la definizione. Ehi! Prendiamo la metà
di quel che ha detto lui, $0.05$ (mezzo decimo): dobbiamo mostrare la parte tra parentesi graffe dell'eq. \ref{definizionelimite}:

\begin{equation}
|x-2|<\nu_\epsilon \Longrightarrow  |\frac{x^2-4}{x-2}-4|<\epsilon  \\
|x-2|<0.05 \Longrightarrow  |\frac{x^2-4}{x-2}-4|=|x-2|<0.1 
\end{equation}

Bè, chiamando $|x-2|$ col nome 'Pippo', è dannatamente ovvio che se Pippo è minore di $0.05$, allora Pippo sarà anche minore
di $0.1$! C'è però un problema: io vorrei andare a casa prima di sera, e invece con questa tecnica devo aspettare che Tommaso
si stanchi di sparare numeri sempre più piccoli! Lui potrebbe dire $0.01$, $0.0000001$, $\pi/10^{42}$, e io dovrei sempre andare
avanti a dimostrare che so trovare un $\nu_\epsilon$ che freghi il suo $\epsilon$. Dobbiamo essere più furbi di così, dobbiamo
inventarci una funzione (che con molta fantasia chiameremo $\nu(\epsilon)$ che ad ogni numero che ci dia Tommaso associ un numero
buono per fregarlo. Proviamo con $\nu(\epsilon) \doteq \frac{\epsilon}{2}$. Applichiamo la definizione:

\begin{equazione}
|x-2|<\nu(\epsilon) \Longrightarrow  |x-2|<\epsilon  \\
|x-2|<\frac{\epsilon}{2} \Longrightarrow  |x-2|<\epsilon 
\end{equazione}

Direi che funziona: se Pippo è minore della metà di $\epsilon$, a maggior ragione è minore di $\epsilon$ (attenti, questo vale
solo perché $\epsilon>0$!!! Non sottovalutate i cavilli). D'ora in poi, possiamo mettere una segreteria telefonica, e ogni volta
che Tommaso telefona gli risponderà: "Quello che hai detto diviso due!", e così potremo anche uscire di casa a farci una birra, ogni tanto. 

Attenzione, questa volta l'uccellino ha suggerito giusto (che $f=4$), ma se ci avesse detto $5$?!? Proviamo, tanto per vedere.
\begin{equazione}
\forall \epsilon>0 \exists \nu_\epsilon : \{ |x-2|<\nu_\epsilon \Longrightarrow  |\frac{x^2-4}{x-2}-5|<\epsilon   \}\\
|\frac{x^2-4-5(x-2)}{x-2}|<\epsilon \longrightarrow |\frac{x^2-4x+4}{x-2}|<\epsilon\\
|\frac{x^2-5x+6}{x-2}|<\epsilon \longrightarrow |\frac{(x-2)(x-3)}{x-2}|<\epsilon\\
|x-3|<\epsilon
\end{equazione}

Adesso, Tommaso - che si è fatto più gentile - ci dice: $\epsilon=10$. Noi rispondiamo, ad esempio (tiro a caso), $\nu=1$. Vediamo se funziona: 
\begin{equazione}
|x-2|<1 \Longrightarrow  |\frac{x^2-4}{x-2}-5|=|x-3|<10 
\end{equazione}
Direi che ci siamo: la parte di sinistra dice che $x$ sta nell'intervallo $[1,3]$ e per qualunque di questi valori direi che $|x-3|$
sta abbondantemente sotto a $10$. Ma aspettate a cantar vittoria: Tommaso ci propone $\epsilon=0.1$. Noi prendiamo un $\nu$ piccolissimo,
diciamo $0.001$. Vediamo che succede:
\begin{equazione}
|x-2|<0.001 \Longrightarrow  |x-3|< 0.1
\end{equazione}
Questa volta, $x \in [1.999,2.001]$; è forse vero che per qualunque valore nel range dato $x$ dista da $3$ non più di $0.1$?!? Certamente
no: la distanza va da $1.001$ a $0.999$ e in ogni caso è ben più grande del valore datoci di Tommaso! Siamo stati fregati! L'uccellino non
ci ha detto la verità!

Dagli esempi visti, siamo stati in grado di dire che il limite $e' 4$ e che il limite $non e' 5$, ma non abbiamo imparato a fare due cose:
(1) a sapere che se è $4$ non può essere nient'altro; (2) a scoprire che fa $4$ senza l'aiuto dell'uccellino. La prima prendetelo come atto
di fiducia, la sezonda sarà scopo del paragrafo successivo.

\subsection{Limiti sinistri e destri}

{\bf Nota:} In realtà, per ogni funfione $f$ e punto $x_0$ esistono {\em due} limiti, detti sinistro e destro; questo è molto
importante poiché non sempre i due limiti coincidono (spesso sì, comunque). Vediamo la definizione di limite sinistro:

\begin{equation}
\lim_{x \to x_0^-} f(x) = y_0
\end{equation}
{\em (che si legge 'limite per $x$ che tende a $x_0$ di $f(x)$ è $y_0$')} se vale la seguente condizione:
\begin{equation}\label{definizionelimitebis}
\forall \epsilon >0 \exists \nu_\epsilon : \{ x<x_0 \wedge |x-x_0|<\nu_\epsilon \Longrightarrow  |f(x)-y_0|<\epsilon   \}
\end{equation}

Come semplice esercizio scrivete anche la definizione di limite destro.

Una funzione $f$ si dice avere limite in un punto $x_0$ se (1) esistono i limiti destro e sinitro e (2) coincidono.


\subsection{Limiti e infinito}

Nell'ambito dei limiti, l'infinito ha senso di esistere come non mai... la definizione di limite cambia impercettibilmente perche'
l'infinito non puo' essere compreso tra due valori $ \pm  \varepsilon $ e quindi ci limitiamo a prendere un valore sinistro (che invece
che indefinitamente piccolo, diventa indefinitamente grande):


\begin{equation}
    \lim_{x \to \infty} f(x) = y_0
\end{equation}

{\em (che si legge 'limite per $x$ che tende a infinito di $f(x)$ è $y_0$')} se vale la seguente condizione:

\begin{equation}\label{definizionelimitebis}
    \forall \epsilon >0 \exists \nu_\epsilon : \{  x > \nu_\epsilon \Longrightarrow  |f(x)-y_0|<\epsilon \}
\end{equation}

Notate che nulla e' cambiato nel codominio, ma solo nel dominio: la $x$, invece di essere compresa tra due carabinieri,
e' invece costretta a destra di un numero arbitrariamente grande.

\section{Limiti notevoli}

\begin{tabular}{|r|l|l|}
\hline
$x_0$ & $f(x)$ & $\lim_{x \rightarrow x_0} f(x)$\\
\hline
\hline
$0$ 		& $x^n$ & $0$ \\
$1$ 		& $x^n$ & $1$ \\
$+\infty$ 	& $x^n$ & $+\infty$\\
\hline

% $\frac{1}{x^n}$ & $\frac{-1}{x^{n+1}}^\dagger$ \\
% $\sqrt{x}$ & $\frac{1}{2\sqrt{x}}^\dagger$ \\
% $\sqrt[n]{x} (=x^{\frac{1}{n}})$ & $\frac{1}{n\sqrt[n]{x^{n-1}}}^\dagger$ \\
% \hline
% $\log_a(x)$ & $\frac{\log_a(e)}{x}$ \\
% $\ln(x)$ & $\frac{1}{x} ^\dagger$ \\
% $\ln|x|$ & $\frac{1}{x} ^\dagger$ \\
% \hline
% $\sin(x)$ & $\cos(x)$ \\
% $\cos(x)$ & $-\sin(x)$ \\
% $\tan(x)$ & $\frac{1}{\cos^2(x)} (=1+\tan^2(x))$ \\
% \hline
% $\arcsin(x)$ & $\frac{1}{\sqrt{1-x^2}}$ \\
% $\arccos(x)$ & $\frac{-1}{\sqrt{1-x^2}}$ \\
% $\arctan(x)$ & $\frac{1}{x^2+1}$ \\
% \hline
% $a^x$ & $a^x \ln a$ \\
% $e^x$ & $e^{x \dagger}$\\
% \hline
% $\sinh(x)$ & $\cosh(x)^\dagger$ \\
% $\cosh(x)$ & $-\sinh(x)^\dagger$ \\
% $\tanh(x)$ & $\frac{1}{\cosh^2(x)}^\dagger$ \\
% \hline
% $arcsinh(x) (= ln|x + \sqrt{x^2+1}|)$ & $\frac{1}{\sqrt{1-x^2}} ^\dagger$ \\
% $arccosh(x) (= ln|x \pm \sqrt{x^2-1}|)$ & $\frac{-1}{\sqrt{1-x^2}} ^\dagger$ \\
% $arctanh(x) (= \frac{1}{2} ln|\frac{1+x}{1-x}|)$ & $\frac{1}{x^2-1} ^\dagger$ \\
\hline
\end{tabular}

\section{Limiti: uso pratico}

Il modo migliore per affrontare i limiti è di \emph{non} sfruttare la definizione (che serve solo per le interrogazioni) ma adottare
i trucchettini che v'insegnerò, quasi fossero dogmi (mica li ho inventati io, sia chiaro!).

Una volta trovati i limiti dei cosiddetti 'mattoncini', potrete usarli liberamente per trovare i limiti fi unzioni ben più complesse.

Trattiamo tutti i limiti 'notevoli' come esercizi, e poi mettiamoli alla fine in una tabella riassuntiva (che deprecabilmente
fotocopierete e lillipuzianamente metterete negli astucci, ...).

\begin{esercizio} Ecco a voi alcuni esercizi da risolvere:
    
    $\lim_{x \rightarrow 0} log(x)= $ ??


    $\lim_{x \rightarrow 1} \frac{1}{x-1}= $ ??


    $\lim_{x \rightarrow 0} \frac{sin(x)}{x}= $ ??

\end{esercizio}


%\subsection{Trucchi coi limiti}
% TODO(ricc): Hospital , deriva su e giù.

