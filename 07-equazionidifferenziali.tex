\label{Equazioni differenziali}
\chapter{Equazioni differenziali}

Questo capitolo ha come prerequisito che conosciate {\em alla perfezione} derivate e integrali.
Non dovete solo averle studiate, ma dovete a sapere a memoria tutte le derivate e integrali notevoli. Se no, fidatevi, non ne uscirete pi\\'u.

\begin{prerequisito}
Qual \'e quella funzione uguale alla propria derivata? e qual \'e quella funzione che derivata una volta cambia di segno?
Cosa si ottiene derivando $n$ volte un innocente $\sin(x)$? Quali famiglie di funzioni derivate un po' di volte e sommate tra loro possono dare una polinomiale fratta (vi aiuto, sono 3)?
\end{prerequisito}


\section{Introduzione}

%{\em "E Dio creo il mondo secondo la legge $-mx''-px'-kx+F(x)=0$ e vide che non era cosa buona; allora cambio qualche segno e vide che era cosa buona..."}


